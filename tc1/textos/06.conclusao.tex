\chapter{Considerações Finais}
\label{conclusao}

O presente trabalho buscou projetar um sistema de gerenciamento de senhas seguro utilizando autenticação biométrica e algoritmos de criptografia. A partir da revisão bibliográfica foi estudado o estado da arte da autenticação biométrica, e identificar quais são suas vantagens e também suas limitações. Também foram pesquisados trabalhos relacionados ao reconhecimento biométrico aplicado em sistemas de gerenciamento de senhas.

Para atingir este objetivo foram estudados trabalhos relacionados onde foram aplicados técnicas de reconhecimento biométrico sobre diversos tipos de biometria. Foram levantados as biometrias e técnicas de reconhecimento utilizadas em cada trabalho e foram comparados seus resultados a fim de definir as biometrias e técnicas a serem aplicadas neste trabalho. Então para o desenvolvimento do sistema de gerenciamento de senhas, foram exploradas duas características biométricas que podem ser obtidas através de dispositivos móveis e que possuem bons resultados já comprovados, que são: faces e voz. 

A primeira parte no desenvolvimento deste trabalho foi a modelagem do sistema de gerenciamento de senhas. Inicialmente foi definida a arquitetura, plataforma alvo e tecnologias envolvidas. Baseando-se nas biometrias, técnicas de reconhecimento e algoritmos de criptografia estudados foram definidos os requisitos funcionais e não funcionais do sistema, e posteriormente foi feito o mapeamento dos principais processos envolvendo a autenticação biométrica e criptografia das senhas armazenadas.

O próximo passo será desenvolver o sistema proposto, no trabalho de conclusão II. Durante o desenvolvimento deverão ser feitos os ajustes necessários para que o sistema possa ser finalizado. Após a conclusão, serão feitos testes sobre o processo de autenticação biométrica, buscando por pontos que podem ser otimizados que não foram considerados durante a modelagem. Com isso, espera-se alcançar todos os requisitos definidos, além de um sistema robusto, seguro, e conveniente para os usuários.

%todo---cronograma do TC2