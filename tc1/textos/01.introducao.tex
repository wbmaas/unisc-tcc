\chapter{Introdução}
\label{introducao}

O uso de senhas para autenticação é um dos meios mais adotados atualmente, estando presente na maioria dos serviços disponíveis na web, e a utilização de uma única senha para diversos serviços pode trazer vulnerabilidades ao usuário, pois a perda ou descoberta desta senha pode comprometer a segurança de todos os seus sistemas e serviços. Por outro lado, o uso de senhas diferentes para cada serviço acaba se tornando inconveniente pois são difíceis de memorizar \cite{Huiguang:2014}.

A autenticação através de biometria estabelece a identidade baseada em características físicas e comportamentais (como face e voz), diminuindo a inconveniência para usuários de terem de criar e lembrar de senhas seguras \cite{gofman:2016}. Infelizmente algumas destas características podem ser facilmente obtidas: faces são publicamente disponíveis e digitais permanecem em superfícies intactas. Por isto, o uso de múltiplas características biométricas é preferido a fim de aumentar a segurança e identificação correta do usuário \cite{Agholor:2016}.

As características biométricas de uma pessoa podem ser reconhecidas através do uso de técnicas de processamento de imagens e voz aliadas à métodos de aprendizado de máquina. Para isto é necessário construir uma base de dados com exemplos de dados obtidos a partir da leitura destas características, sobre o qual são aplicados os algoritmos de aprendizado de máquina a fim de construir um modelo das propriedades biométricos de cada pessoa \cite{heinen:2005}.

Segundo \cite{mitchell:l997}, um programa aprende quando sua performance melhora com a experiência em uma determinada tarefa. Elas são especialmente úteis em domínios onde humanos ainda não possuem o conhecimento necessário para desenvolver algoritmos efetivos, como reconhecimento de faces humanas em imagens. 

Técnicas de aprendizado de máquina têm sido utilizadas por obterem ótimos resultados em uma grande variedade de aplicações. Em seu trabalho \cite{violajones:2001}, demonstra que é possível obter ótimos resultados na detecção de faces em imagens, utilizando árvores de decisão e grandes bases de treinamento, chegando a taxas de reconhecimento acima de 94\%.

Já \cite{ranny:2016} em seu trabalho sobre reconhecimento de voz, atinge uma taxa de reconhecimento de 84.5\%, através do algoritmo \textit{k-Nearest Neighbors}, e surpreendentes 96.97\% utilizando o método de distância dupla.

O reconhecimento da íris também demonstra grande potencial para autenticação biométrica, pois trabalhos como o de \cite{pavaloi:2017} demonstram ótimos resultados neste tipo de problema, utilizando as técnicas de \textit{Support Vector Machine} e \textit{k-Nearest Neighbors}, e bases de treinamento como UBIRIS \cite{ubiris:2009}, o autor obteve taxas de reconhecimento acima de 90\%.

Entretanto o uso de autenticação biométrica em um sistema de gerenciamento de senhas melhora sua usabilidade e segurança, porém para a garantir a confiabilidade geral do sistema é preciso realizar o armazenamento de senhas de uma forma segura, pois o armazenamento em formato de texto simples é uma solução extremamente frágil em termos de segurança. Para isto podem ser utilizadas técnicas de criptografia.

Segundo \cite{schneier:2007} um algoritmo de criptografia, também conhecido como cifrador, é uma função matemática usada para cifragem e decifragem. Para \cite{stallings:2014} algoritmos criptográficos são técnicas para garantir o sigilo e/ou a autenticidade da informação. Os dois ramos principais da criptologia são a criptografia, que é o estudo do projeto dessas técnicas; e a criptoanálise, que trata de frustrar essas técnicas, recuperar informações ou forjar informações que serão aceitas como autênticas.

Neste trabalho será explorado o uso de algoritmos de criptografia baseados em chaves, que podem ser classificados em dois tipos: simétricos, onde geralmente existe uma chave em comum que é utilizada para cifragem e decrifragem; e os de chave pública, onde a chave de cifragem é diferente da de decifragem \cite{schneier:2007}. 

Em seu trabalho, \cite{panda:2016} realiza diversos testes sobre os principais algoritmos de criptografia simétricos e assimétricos, como: AES, DES, RSA, BLOWFISH. E ele observa que o desempenho de algoritmos assimétricos é quase mil vezes mais lento quando comparados aos simétricos, em relação ao seu tempo de cifragem, decifragem e vazão.

A vantagem do armazenamento de senhas criptografadas é que mesmo que o acesso a elas seja comprometido elas não podem ser facilmente recuperadas sem que se tenha a chave utilizada na encriptação, e os dados obtidos não passam de sequencias aleatórias de caracteres \cite{chanda:2016}.

