\begin{abstract}
\keywords{Gerenciador de Senhas, Autenticação Biométrica, Aprendizado de máquina, Criptografia}
  
Milhares de serviços são oferecidos na \textit{web}, e a grande maioria destes serviços utilizam alguma forma de autenticação, geralmente através de usuário e senha. Entretanto, o uso de senhas pode tornar a autenticação frágil quando estas forem muito simples, trazendo pouca segurança. Por outro lado, quando forem complexas e muito grandes são difíceis de memorizar e os usuários muitas vezes guardam anotações destas em locais inseguros, tornando o processo frágil. Desta forma, sistemas de gerenciamento de senhas podem ser uma solução para evitar o uso de senhas fracas e repetidas em diferentes serviços sem a necessidade de serem memorizadas pelo usuário (ou anotadas). No entanto, é necessário utilizar métodos eficientes de autenticação nesses sistemas, pois uma vez que alguém consegue acessá-los, todas as senhas salvas podem ser vistas. Assim, uma boa alternativa é a autenticação biométrica utilizando características como faces e voz, que pode proporcionar uma camada extra de segurança, enquanto melhora a conveniência no seu uso. Técnicas de aprendizado de máquina aplicadas em características biométricas possuem altas taxas de reconhecimento, e estes resultados podem ser combinados entre diferentes biometrias para melhorar a segurança no processo de autenticação. A tecnologia presente nos dispositivos móveis atuais permite o desenvolvimento de sistemas que sejam capazes de processar características biométricas com poucos recursos computacionais, e possuem os sensores necessários para o processo de coleta de amostras das mesmas, através de câmeras, microfones e leitores de digitais. Este trabalho apresenta o projeto de um sistema de gerenciamento de senhas seguro utilizando criptografia e reconhecimento de biometrias de face e voz.


%%todo - - eu alterei bastante o resumo. Veja se está coerente. E precisa adicionar uma frase importante no final, informando o que esse trabalho faz...por ex.: "Este trabalho apresenta ..." O resumo sempre deve mostrar que tem um problema, que ele pode ser solucionado usando alguma coisa, e que esse trabalho vai mostrar o desenvolvimento de uma solução....quando já se tem resultados, deve-se mencionar eles (por ex., os resultados mostram blablabla" PS: O resumo é importantíssimo em um trabalho, pois muitas vezes o leitor lê apenas ele e decide se vale a pena ler o trabalho todo ou não.
\end{abstract}
