\documentclass[propostatc]{unisc}    % Opção showlinks exibe os links em cor azul

\usepackage[T1]{fontenc}                       % Suporte a acentuação no arquivo de saída
\usepackage[utf8]{inputenc}                    % Codificação dos arquivos de entrada em UTF-8
\usepackage[english,brazilian]{babel}          % Idiomas utilizados no trabalho (o último é o principal)
\usepackage{float}                             % Maior controle de objetos "float" (tabelas, figuras, etc.)
\usepackage{enumerate}
\usepackage{graphicx}



\title{Sistema de armazenamento de senhas seguro com autenticação biométrica}    % Título do Trabalho
\author{Maas}{William Bozzetti}             % Autor do Trabalho
\advisor[Prof. Me.]{Neu}{Charles Varlei}    % Orientador
\reviewer[Prof. Dra.]{Frozza}{Rejane}       % Avaliador 1
\reviewer[Prof. Me.]{Helfer}{Gilson}        % Avaliador 2
\reviewer[Prof. Dr.]{Tedesco}{Leonel}       % Avaliador 3

\dept{Departamento de Informática}
\course{Curso de Ciência da Computação}
\location{Santa Cruz do Sul}{RS}
\date{22}{agosto}{2017}


\begin{document}

\maketitle
\tableofcontents

\chapter*{Introdução}

O uso de senhas para autenticação é um dos meios mais adotados atualmente, estando presente na maioria dos serviços disponíveis na web, e a utilização de uma única senha para diversos serviços pode trazer vulnerabilidades ao usuário, pois a perda ou descoberta desta senha pode comprometer a segurança de todos os seus sistemas e serviços. Por outro lado, o uso de senhas diferentes para cada serviço acaba se tornando inconveniente pois são difíceis de memorizar \cite{Huiguang:2014}.

A autenticação através de biometria estabelece a identidade baseada em características físicas e comportamentais (como face e voz), diminuindo a inconveniência para usuários de terem de criar e lembrar de senhas seguras \cite{gofman:2016}. Infelizmente algumas destas características podem ser facilmente obtidas: faces são publicamente disponíveis e digitais permanecem em superfícies intactas. Por isto, o uso de múltiplas características biométricas são preferidos a fim de aumentar a segurança e identificação correta do usuário \cite{Agholor:2016}.

As características biométricas de uma pessoa podem ser reconhecidas através do uso de técnicas de processamento de imagens e voz aliadas à métodos de aprendizado de máquina. Para isto é necessário construir uma base de dados com exemplos de dados obtidos a partir da leitura destas características, sobre o qual são aplicados os algoritmos de aprendizado de máquina a fim de construir um modelo das propriedades biométricos de cada pessoa \cite{heinen:2005}.

Segundo \cite{Mitchell:l997}, um programa aprende quando sua performance melhora com a experiência em uma determinada tarefa. Elas são especialmente úteis em domínios onde humanos ainda não possuem o conhecimento necessário para desenvolver algoritmos efetivos, como reconhecimento de faces humanas em imagens. 

Técnicas de aprendizado de máquina têm sido utilizadas por obterem ótimos resultados em uma grande variedade de aplicações. Em seu trabalho \cite{violajones:2001}, demonstra que é possível obter ótimos resultados na detecção de faces em imagens, utilizando árvores de decisão e grandes bases de treinamento, chegando a taxas de reconhecimento acima de 94\%.

Já \cite{ranny:2016} em seu trabalho sobre reconhecimento de voz, atinge uma taxa de reconhecimento de 84.5\%, através do algoritmo \textit{k-Nearest Neighbors}, e surpreendentes 96.97\% utilizando o método de distância dupla.

O reconhecimento da íris também demonstra grande potencial para autenticação biométrica, pois trabalhos como o de \cite{pavaloi:2017} demonstram ótimos resultados neste tipo de problema, utilizando as técnicas de \textit{Support Vector Machine} e \textit{k-Nearest Neighbors}, e bases de treinamento como UBIRIS \cite{ubiris:2009}, o autor obteve taxas de reconhecimento acima de 90\%.

Entretanto o uso de autenticação biométrica em um sistema de gerenciamento de senhas melhora sua usabilidade e segurança, porém para a garantir a confiabilidade geral do sistema é preciso realizar o armazenamento de senhas de uma forma segura, pois o armazenamento em formato de texto simples é uma solução extremamente frágil em termos de segurança. Para isto podem ser utilizadas técnicas de criptografia.

Segundo \cite{schneier:2007} um algoritmo de criptografia, também conhecido como cifrador, é uma função matemática usada para cifragem e decifragem. Para \cite{stallings:2014} algoritmos criptográficos são técnicas para garantir o sigilo e/ou a autenticidade da informação. Os dois ramos principais da criptologia são a criptografia, que é o estudo do projeto dessas técnicas; e a criptoanálise, que trata de frustrar essas técnicas, recuperar informações ou forjar informações que serão aceitas como autênticas.

Neste trabalho será explorado o uso de algoritmos de criptografia baseados em chaves, que podem ser classificados em dois tipos: simétricos, onde geralmente existe uma chave em comum que é utilizada para cifragem e decrifragem; e os de chave pública, onde a chave de cifragem é diferente da de decifragem \cite{schneier:2007}. 

Em seu trabalho, \cite{panda:2016} realiza diversos testes sobre os principais algoritmos de criptografia simétricos e assimétricos, como: AES, DES, RSA, BLOWFISH. E ele observa que o desempenho de algoritmos assimétricos é quase mil vezes mais lento quando comparados aos simétricos, em relação ao seu tempo de cifragem, decifragem e vazão.

A vantagem do armazenamento de senhas criptografadas é que mesmo que o acesso a elas seja comprometido elas não podem ser facilmente recuperadas sem que se tenha a chave utilizada na encriptação, e os dados obtidos não passam de sequencias aleatórias de caracteres \cite{chanda:2016}.

Para a proposta deste trabalho serão estudados diferentes algoritmos de aprendizado de máquina analisando suas vantagens na aplicação do reconhecimento de faces, íris, digitais e voz. posteriormente será feito o uso destas técnicas para realizar autenticação no sistema de gerenciamento de senhas desenvolvido. Também serão utilizados algoritmos de criptografia para o armazenamento das senhas de forma segura no sistema. As principais etapas de como esse processo é realizado são apresentadas na Figura \ref{fig:imagem}.

\begin{figure}[!htb]
  \centering
  \caption{Visão geral do sistema de gerenciamento de senhas.}
  \includegraphics[width=\textwidth]{Imagem.png}
  \fonte{Elaborado pelo autor.}
  \label{fig:imagem}
\end{figure}

O fluxo do sistema será de forma simples, primeiro serão coletadas as leituras de pelo menos duas características biométricas dos usuários através do uso de câmera, microfone, e leitor de digitais (que serão definidas durante o trabalho). Estas informações serão processadas através do uso de algoritmos de aprendizado de máquina, onde a combinação dos resultados irá definir se a autenticação é válida ou não. Sendo autenticado, o usuário irá ter acesso ao sistema que permitirá realizar a leitura, cadastro e alteração de informações de login como usuário e/ou email, descrição(nome ou URL do sistema), e senha.

O sistema também contará com funções para gerar senhas automaticamente, baseando-se em boas práticas de segurança; escolha entre algoritmos de criptografia e definição da chave de cifragem. O \textit{software} desenvolvido será para a plataforma web (e/ou mobile), e os dados serão armazenados em um banco de dados.



\chapter*{Justificativa}
%%Justificativa Social
Como justificativa social para o desenvolvimento deste tema, o uso de técnicas de aprendizado de máquina para reconhecimento de características biométricas implicam no aumento da segurança e melhora na usabilidade de um sistema de gerenciamento de senhas, removendo a exigência de memorização de uma senha principal para acesso ao sistema, o que pode ocasionar a popularização do uso deste tipo de sistema. E o uso de técnicas de criptografia proveem proteção extra em caso de comprometimento dos dados, pois eles não poderão ser facilmente decifrados sem as chaves utilizadas na cifragem.

%%Justificativa Cientifica
Como justificativa científica, a área de aprendizado de máquina aplicada sobre reconhecimento de características biométricas pode trazer contribuições significativas, tanto como demonstrar a viabilidade ou não destas características no uso de autenticação, quanto verificar o desempenho das técnicas de aprendizado de máquina aplicadas sobre o reconhecimento de padrões.

%%Justificativa Empresarial
Como justificativa empresarial, o aumento do número de dispositivos com câmeras de alta qualidade, leitores de digitais, entre outros, permite que sistemas façam proveito destas tecnologias, para que os consumidores tenham interesse em adquiri-los, e para que a indústria continue a evoluir sobre estas tecnologias.

Como problema de pesquisa, define-se: "É possível desenvolver um sistema de armazenamento seguro de senhas que somente podem ser acessadas através de autenticação biométrica, utilizando técnicas de ML e criptografia?".


\chapter*{Objetivos}
O objetivo geral deste trabalho é desenvolver um sistema de armazenamento seguro de senhas com autenticação biométrica. Pretende-se aplicar técnicas de aprendizado de máquina para realizar a autenticação biométrica e algoritmos de criptografia para armazenar as senhas.

Sendo assim, pretende-se atingir os seguintes objetivos específicos:

\begin{itemize}
\item Estudar trabalhos relacionados ao reconhecimento de características biométricas.
\item Estudar trabalhos relacionados ao uso de criptografia.
\item Estudar os principais métodos de criptografia.
\item Estudar técnicas de processamento de imagens, reconhecimento e aprendizado de máquina e sua aplicação sobre reconhecimento de características biométricas como: faces, íris, digitais e voz.
\item Modelar o sistema de gerenciamento de senhas aplicando as técnicas estudadas.
\item Desenvolver o sistema de gerenciamento de senhas.
\end{itemize}


\chapter*{Metodologia}
Esta Seção apresenta a caracterização da pesquisa e os procedimentos metodológicos aplicados a este trabalho.

\section*{Caracterização da pesquisa}
O objetivo desta pesquisa é exploratório. Serão pesquisados trabalhos relacionados a aplicação de técnicas de aprendizado de máquina no reconhecimento de características biométricas para autenticação em gerenciadores de senhas, e técnicas de criptografia. O ambiente de pesquisa é bibliográfico, onde serão buscadas fontes através de livros sobre segurança e criptografia de dados, autenticação utilizando características biométricas, aprendizado de máquina, artigos, outros trabalhos de conclusão desenvolvidos, Internet.

O procedimento de validação do sistema será qualitativo, onde será feita uma síntese dos resultados obtidos, evidenciando as contribuições, relatando as limitações do estudo, relacionando os fatos verificados com a teoria. Além disso, serão feitos testes sobre o processo de autenticação através do uso de biometria e algoritmos de aprendizado de máquina.

\section*{Procedimento metodológicos}
Para a realização do trabalho de conclusão apresentado serão necessárias as seguintes etapas:

\begin{enumerate}[I.]
  \item Elaboração da proposta
  \begin{itemize}
    \item Realizar o levantamento bibliográfico inicial sobre assuntos a serem trabalhados no TC.
    \item Escrita da proposta.
    \item Entrega da proposta.
  \end{itemize}
  \item Pesquisa sobre o estado da arte, estudando trabalhos semelhantes da Literatura.
  \item Estudos e definições.
  \begin{itemize}
    \item Estudo sobre algoritmos de criptografia.
    \item Estudo sobre o uso de características biométricas para autenticação.
    \item Estudo sobre reconhecimento de faces, digitais, íris e voz.
    \item Estudo sobre técnicas de aprendizado de máquina aplicados sobre características biométricas.
    \item Definição de duas ou mais características biométricas a serem implementadar no TC.
    \item Definição das técnicas de aprendizado de máquina utilizadas na autenticação biométrica.
    \item Definição das técnicas de criptografia a serem utilizadas no armazenamento de senhas.
    \item Estudo sobre bibliotecas que implementam técnicas de aprendizado de máquina e processamento de dados e imagens como: scikit-learn \cite{scikit-learn}, scikit-image \cite{scikit-image}, OpenCV \cite{opencv}.
    \item Estudo sobre bibliotecas que implementam algoritmos de criptografia como: Pycrypto.
  \end{itemize}
  \item Modelagem do sistema de gerenciamento de senhas.
  \begin{itemize}
    \item Diagrama de classes.
    \item Diagrama ER.
    \item Diagrama de casos de uso.
    \item Diagrama de sequência.
  \end{itemize}
  \item Escrita do TCI.
  \item Entrega do TCI.
  \item Implementação do sistema de gerenciamento de senhas utilizando as técnicas de aprendizado de máquina e algoritmos de criptografia estudados.
  \begin{itemize}
    \item Implementar o reconhecimento biométrico "1".
    \item Implementar o reconhecimento biométrico "2".
    \item Implementar a criptografia.
    \item Implementar a criptografia sobre os dados que serão armazenados.
    \item Implementar o armazenamento, consulta e alteração dos dados.
  \end{itemize}
  \item Treinamento com bases de dados de características biométricas como: SCFace \cite{grgic:2011}, UBIRIS \cite{ubiris:2009}, entre outros.
  \item Realização de testes e análise dos resultados.
  \item Escrita do TCII.
  \item Apresentação no Salão de Iniciação Científica da UNISC.
  \item Entrega do TCII.
  \item Defesa do Trabalho de Conclusão.
\end{enumerate}


\chapter*{Cronograma}
A seguir, apresenta-se o cronograma para desenvolvimento do trabalho de conclusão.

\begin{table}[H]
  \label{tab:cronograma}
  \centering
  \begin{tabular}{c|c|c|c|c|c|c|c|c|c|c|c|c|c|}
    \cline{2-14} & \multicolumn{6}{c|}{2017} & \multicolumn{7}{c|}{2018} \\
    \cline{2-14} & Jul & Ago & Set & Out & Nov & Dez & Jan & Fev & Mar & Abr & Mai & Jun & Jul \\
    %                                       JUL AGO SET OUT NOV DEZ JAN FEV MAR ABR MAI JUN JUL
    \cline{1-14} \multicolumn{1}{|c|}{I}   & X & X &   &   &   &   &   &   &   &   &   &   &   \\
    \cline{1-14} \multicolumn{1}{|c|}{II}  &   & X & X &   &   &   &   &   &   &   &   &   &   \\
    \cline{1-14} \multicolumn{1}{|c|}{III} &   & X & X & X & X &   &   &   &   &   &   &   &   \\
    \cline{1-14} \multicolumn{1}{|c|}{IV}  &   &   &   &   & X & X &   &   &   &   &   &   &   \\
    \cline{1-14} \multicolumn{1}{|c|}{V}   &   &   & X & X & X & X &   &   &   &   &   &   &   \\
    \cline{1-14} \multicolumn{1}{|c|}{VI}  &   &   &   &   &   & X &   &   &   &   &   &   &   \\
    \cline{1-14} \multicolumn{1}{|c|}{VII} &   &   &   &   &   &   & X & X & X & X & X &   &   \\
    \cline{1-14} \multicolumn{1}{|c|}{VIII}&   &   &   &   &   &   &   &   & X & X & X &   &   \\
    \cline{1-14} \multicolumn{1}{|c|}{IX}  &   &   &   &   &   &   &   &   &   & X & X &   &   \\
    \cline{1-14} \multicolumn{1}{|c|}{X}   &   &   &   &   &   &   &   &   & X & X & X & X &   \\
    \cline{1-14} \multicolumn{1}{|c|}{XI}  &   &   &   &   &   &   &   &   &   &   &   & X &   \\
    \cline{1-14} \multicolumn{1}{|c|}{XII} &   &   &   &   &   &   &   &   &   &   &   & X &   \\
    \cline{1-14} \multicolumn{1}{|c|}{XIII}&   &   &   &   &   &   &   &   &   &   &   &   & X \\
    \cline{1-14}
  \end{tabular}
\end{table}

% Carrega as referências do arquivo referencias.bib
\bibliographystyle{abntex2-alf}
\bibliography{referencias}

% Opcional, dependendo do orientador
\makesignature

\end{document}
