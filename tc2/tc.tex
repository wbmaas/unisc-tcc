\documentclass[tc]{unisc}

\usepackage[T1]{fontenc}        % Suporte a acentuação no arquivo de saída
\usepackage[utf8]{inputenc}     % Codificação dos arquivos de entrada em UTF-8
\usepackage[english,brazilian]{babel}

\usepackage{graphicx}           % Para adicionar figuras
\usepackage{float}              % Maior controle de objetos "float" (tabelas, figuras, etc.)
\usepackage{enumerate}

\usepackage{capt-of}
\usepackage{longtable}
\usepackage{tabularx}

\graphicspath{ {images/} }


% ==================================================================================
%
%                         INFORMAÇÕES GERAIS
%
% ==================================================================================

%\title{Sistema de armazenamento de senhas seguro com autenticação biométrica}    % Título do Trabalho
\title{PWFaces: Gerenciador de senhas com autenticação biométrica}
\author{Maas}{William Bozzetti}             % Autor do Trabalho
\advisor[Prof. Me.]{Neu}{Charles Varlei}    % Orientador
%\reviewer[Prof. Dra.]{Frozza}{Rejane}       % Avaliador 1
\reviewer[Prof. Me.]{Helfer}{Gilson}        % Avaliador 2
\reviewer[Prof. Dr.]{Tedesco}{Leonel}       % Avaliador 3

\dept{Departamento de Informática}
\course{Curso de Ciência da Computação}
\degree{Bacharel em Ciência da Computação}
\location{Santa Cruz do Sul}{RS}
\date{22}{junho}{2018}

\makeindex

% ==================================================================================
%
%                         CONTEÚDO
%
% ==================================================================================


\begin{document}

\makecapa
\maketitle

\begin{abstract}
\keywords{Gerenciamento de Senhas, Autenticação Biométrica, Criptografia}
  
Milhares de serviços são oferecidos na \textit{web}, e a grande maioria deles utilizam alguma forma de autenticação, geralmente através de usuário e senha. Entretanto, o uso de senhas pode tornar a autenticação frágil quando estas forem muito simples, trazendo pouca segurança. Por outro lado, quando forem complexas e muito grandes são difíceis de memorizar e os usuários muitas vezes guardam anotações destas em locais inseguros, tornando o processo frágil. Desta forma, sistemas de gerenciamento de senhas podem ser uma solução para evitar o uso de senhas fracas e repetidas em diferentes serviços sem a necessidade de serem memorizadas pelo usuário (ou anotadas). De um lado, é necessário utilizar métodos eficientes de autenticação nesses sistemas, pois uma vez que alguém consegue fraudar esse processo, todas as senhas salvas podem ser comprometidas. Por outro lado, é preciso armazenar as senhas de modo seguro utilizando, por exemplo, técnicas de criptografia. Assim, uma boa alternativa é a autenticação biométrica utilizando características como faces e voz, que pode proporcionar uma camada extra de segurança, enquanto melhora a conveniência no seu uso. A maioria dos dispositivos atuais permite o desenvolvimento de sistemas que sejam capazes de processar características biométricas, pois possuem os sensores necessários para o processo de coleta de amostras, através de câmeras, microfones e leitores de digitais e o poder computacional necessário para processá-las. Este trabalho apresenta um sistema de geração de senhas complexas e armazenamento seguro, utilizando criptografia, e autenticação através de reconhecimento facial.
\end{abstract}

\begin{abstract}[english][Abstract]
\keywords[Keywords]{Password Manager, Biometric Authentication, Encryption}

Thousands of services are offered on the web, and the vast majority of them use some form of authentication, usually through username and password. However, the use of passwords can make authentication fragile when it is very simple, with reduced security. On the other hand, when they are complex and very large they are difficult to memorize and users often keep notes of these in unsafe places, making the process fragile. In this way, password management systems can be a solution to avoid the use of weak and repeated passwords in different services without the need to be memorized or annotated by the user. In one hand, one need to securely store passwords using encryption techniques so that information security is present at all times, no matter where they are stored. There are several techniques and algorithms for data encryption, some are extremely secure and others not so much. On the other hand, it is necessary to use efficient methods of authentication in these systems, because once someone can access them, all saved passwords can be seen. Thus, a good alternative is biometric authentication using features such as faces and voice, which can provide an extra layer of security, while improving the convenience in its use. The technology present in the current devices allows the development of systems that are capable of processing biometric features with few computational resources, and have the necessary sensors for the process of collecting samples of them, through cameras, microphones and digital readers. This work presents the design of a secure password storage and generation system using encryption and biometric authentication.
\end{abstract}
\listoffigures
\listoftables
\tableofcontents

\chapter{Introdução}
\label{introducao}

O uso de senhas para autenticação é um dos meios mais adotados atualmente, estando presente na maioria dos serviços disponíveis na web, e a utilização de uma única senha para diversos serviços pode trazer vulnerabilidades ao usuário, pois a perda ou descoberta desta senha pode comprometer a segurança de todos os seus sistemas e serviços. Por outro lado, o uso de senhas diferentes para cada serviço acaba se tornando inconveniente pois são difíceis de memorizar \cite{Huiguang:2014}.

A autenticação através de biometria estabelece a identidade baseada em características físicas e comportamentais (como face e voz), diminuindo a inconveniência para usuários de terem de criar e lembrar de senhas seguras \cite{gofman:2016}. Infelizmente algumas destas características podem ser facilmente obtidas: faces são publicamente disponíveis e digitais permanecem em superfícies intactas. Por isto, o uso de múltiplas características biométricas é preferido a fim de aumentar a segurança e identificação correta do usuário \cite{Agholor:2016}.

As características biométricas de uma pessoa podem ser reconhecidas através do uso de técnicas de processamento de imagens e voz aliadas à métodos de aprendizado de máquina. Para isto é necessário construir uma base de dados com exemplos de dados obtidos a partir da leitura destas características, sobre o qual são aplicados os algoritmos de aprendizado de máquina a fim de construir um modelo das propriedades biométricos de cada pessoa \cite{heinen:2005}.

Segundo \cite{mitchell:l997}, um programa aprende quando sua performance melhora com a experiência em uma determinada tarefa. Elas são especialmente úteis em domínios onde humanos ainda não possuem o conhecimento necessário para desenvolver algoritmos efetivos, como reconhecimento de faces humanas em imagens. 

Técnicas de aprendizado de máquina têm sido utilizadas por obterem ótimos resultados em uma grande variedade de aplicações. Em seu trabalho \cite{violajones:2001}, demonstra que é possível obter ótimos resultados na detecção de faces em imagens, utilizando árvores de decisão e grandes bases de treinamento, chegando a taxas de reconhecimento acima de 94\%.

Já \cite{ranny:2016} em seu trabalho sobre reconhecimento de voz, atinge uma taxa de reconhecimento de 84.5\%, através do algoritmo \textit{k-Nearest Neighbors}, e surpreendentes 96.97\% utilizando o método de distância dupla.

O reconhecimento da íris também demonstra grande potencial para autenticação biométrica, pois trabalhos como o de \cite{pavaloi:2017} demonstram ótimos resultados neste tipo de problema, utilizando as técnicas de \textit{Support Vector Machine} e \textit{k-Nearest Neighbors}, e bases de treinamento como UBIRIS \cite{ubiris:2009}, o autor obteve taxas de reconhecimento acima de 90\%.

Entretanto o uso de autenticação biométrica em um sistema de gerenciamento de senhas melhora sua usabilidade e segurança, porém para a garantir a confiabilidade geral do sistema é preciso realizar o armazenamento de senhas de uma forma segura, pois o armazenamento em formato de texto simples é uma solução extremamente frágil em termos de segurança. Para isto podem ser utilizadas técnicas de criptografia.

Segundo \cite{schneier:2007} um algoritmo de criptografia, também conhecido como cifrador, é uma função matemática usada para cifragem e decifragem. Para \cite{stallings:2014} algoritmos criptográficos são técnicas para garantir o sigilo e/ou a autenticidade da informação. Os dois ramos principais da criptologia são a criptografia, que é o estudo do projeto dessas técnicas; e a criptoanálise, que trata de frustrar essas técnicas, recuperar informações ou forjar informações que serão aceitas como autênticas.

Neste trabalho será explorado o uso de algoritmos de criptografia baseados em chaves, que podem ser classificados em dois tipos: simétricos, onde geralmente existe uma chave em comum que é utilizada para cifragem e decrifragem; e os de chave pública, onde a chave de cifragem é diferente da de decifragem \cite{schneier:2007}. 

Em seu trabalho, \cite{panda:2016} realiza diversos testes sobre os principais algoritmos de criptografia simétricos e assimétricos, como: AES, DES, RSA, BLOWFISH. E ele observa que o desempenho de algoritmos assimétricos é quase mil vezes mais lento quando comparados aos simétricos, em relação ao seu tempo de cifragem, decifragem e vazão.

A vantagem do armazenamento de senhas criptografadas é que mesmo que o acesso a elas seja comprometido elas não podem ser facilmente recuperadas sem que se tenha a chave utilizada na encriptação, e os dados obtidos não passam de sequencias aleatórias de caracteres \cite{chanda:2016}.


\chapter{Fundamentação Teórica}
\label{fundamentacao-teorica}

Este capítulo aborda conceitos fundamentais para a realização da pesquisa e a descrição das principais técnicas para o desenvolvimento de um sistema de gerenciamento de senhas seguro com autenticação biométrica.



\section{Autenticação Biométrica}
Existem três diferentes métodos para autenticação de usuários que são: \textit{"Algo que você sabe"} como senhas e PINs; \textit{"Algo que você tem"} como cartões físicos e tokens; e finalmente \textit{"Algo que você é"} como características físicas e comportamentais únicas dos seres humanos \cite{aliasgari:2015}.

Autenticação biométrica procura identificar a identidade de uma pessoa através de uma ou mais de características físicas ou comportamentais únicas como: digitais, face, voz e íris. Os dispositivos biométricos capturam as características biométricas dos usuários e transformam em informações digitais para que possa interpreta-las e reconhece-las \cite{Huiguang:2014}.

Segundo \cite{gofman:2016} o uso de múltiplas características físicas ou comportamentais é o próximo passo para sistemas de autenticação biométrica mais seguros e robustos. E com a tecnologia atual presente na grande maioria dos \textit{smartphones} é possível tornar o uso de biometria multimodal conveniente para os usuários do dia a dia.



\section{Reconhecimento de padrões}
Segundo \cite{theodoridis:2009} reconhecimento de padrões é uma disciplina científica onde o objetivo é a classificação de objetos em categorias ou classes. Dependendo da aplicação, estes objetos podem ser imagens ou ondas de sinais ou qualquer tipo de medidas que precisam ser classificadas. 



\subsection{Reconhecimento facial}
Reconhecimento facial é o processo de identificar a face de uma pessoa em uma imagem ou video e este processo é dividido em várias etapas, algumas das principais são: detecção da face na imagem, sua extração, e a comparação entre a entrada e a imagem da face salva anteriormente \cite{rajesh:2016}.

Paul Viola e Michael Jones em 2001 criaram um algoritmo eficiente para realizar a localização de faces em uma imagem. O algoritmo viola-jones consegue detectar objetos em imagens minimizando a computação necessária enquanto atinge resultados em tempo real, e tem seu resultado comprovado na identificação de faces em imagens \cite{violajones:2001}, e ainda é considerado como referência atualmente.

Segundo \cite{opencv} o reconhecimento facial através de características geométricas da face é provavelmente o método mais simples, e ele bastante robusto em mudanças de iluminação, porém o posicionamento de pontos de referência (posição dos olhos, nariz, orelhas) no rosto são difíceis de serem calculados com exatidão, para que se possa formar um vetor de características como distância entre os pontos e angulo entre eles.

Em seu trabalho \cite{belhumeur:1997} realizou uma análise entre quatro métodos utilizados para resolver o problema de reconhecimento de faces, que são: \textit{Correlation}, \textit{Linear Subspaces}, \textit{Eigenfaces} e \textit{Fisherfaces}. Através de testes utilizando imagens em condições de iluminação diferentes, e com expressões faciais diferentes, obteve-se um bom resultado com o método de \textit{Fisherfaces}, chegando a taxas de erro inferiores a 1\%, como pode ser observado na Figura \ref{fig:reconhecimento-facial}.

\begin{figure}[H]
  \centering
  \caption{Taxas de erro entre diferentes métodos de reconhecimento facial.}
  \includegraphics[width=\textwidth]{ReconhecimentoFacial.png}
  \fonte{\cite{belhumeur:1997}}
  \label{fig:reconhecimento-facial}
\end{figure}

Segundo \cite{rajesh:2016} \textit{Fisherfaces} é um dos melhores algoritmos na detecção de faces apresentando uma taxa de sucesso de aproximadamente 96\%. Ele realiza duas classes de análises para alcançar o reconhecimento que são: (PCA) \textit{principal component analisys} e (LDA) \textit{linear discrimination analysis} respectivamente. PCA é usado para reduzir a dimensionalidade da imagem, através deste processo o número de características da imagem também é reduzido. Já o LDA é um método discriminante usado em muitos problemas de reconhecimento, ele computa o grupo de características que normalizam as diferentes classes de dados da imagem para classificação.



\subsection{Reconhecimento de voz}
O reconhecimento de voz é uma das soluções que se desenvolveram baseados no reconhecimento de padrões. O processo de reconhecimento de voz na ciência da computação pode ser alcançado por diversos métodos, como: \textit{Dynamic Time Wrapping} (DWT), \textit{Linear Vector Quantization} (LVQ), \textit{Artificial Neural Networks} (ANN), e etc. Cada método tem suas vantagens e desvantagens baseado no tempo de processamento e precisão no reconhecimento \cite{ranny:2016}.

Em seu trabalho \cite{aliasgari:2015} demonstra que o é possível realizar o reconhecimento de voz com algoritmos de aprendizado de máquina como o \textit{K-Nearest Neighbor} (k-NN). O reconhecimento funciona com a extração de características através da técnica de \textit{cepstral melodic analysis}, então é feita a conversão para escala de mel que é mais apropriada para o reconhecimento da voz, e por fim é calculado o \textit{delta mel frequency cepstral coefficients} (DDMFCC) que pode ser classificado utilizando o k-NN.





\subsection{Reconhecimento de íris}
Segundo \cite{pavaloi:2017} uma maneira comum de autenticação biométrica é através do reconhecimento de íris, uma das mais seguras e eficazes características biométricas. Atualmente existe um grande interesse em utilizar reconhecimento de íris para autenticação em \textit{smartphones} e \textit{tablets}, e este processo é parecido com o reconhecimento de outras características biométricas como faces, e é dividido em duas etapas, na primeira etapa, são adquiridas uma ou mais imagens da íris da pessoa, por meio de câmeras digitais no spectrum visível ou infravermelho. Na segunda etapa as imagens da íris capturadas são comparadas com as imagens armazenadas no banco de dados de íris. Duas imagens de íris são consideradas da mesma classe somente se elas pertencem a mesma pessoa. E através do uso de técnicas de aprendizado de máquina como k-NN (\textit{k - Nearest Neighborhood}) e SVM (\textit{Support Vector Machine}) é possível alcançar uma taxa de aproximadamente 96\% de reconhecimento.



\section{Aprendizado de Máquina}
Segundo \cite{scikit-learn} aprendizado de máquina se trata de aprender propriedades sobre um conjunto de dados, e aplicar este conhecimento sobre novos dados. Uma prática comum para avaliar um algoritmo de aprendizado de máquina é dividir os dados que se tem disponível em dois conjuntos, o conjunto de treinamento que será utilizado para extração de propriedades dos dados, e o conjunto de testes, que é utilizado para testar estas propriedades.

Segundo \cite{george:2004} o reconhecimento de padrões teve suas origens da engenharia, onde o aprendizado de máquina surgiu da ciência da computação. Porém, ambas atividades podem ser vistas como duas faces da mesma área, e juntas elas têm se desenvolvido rapidamente na última década. Para \cite{mitchell:l997}, algoritmos de aprendizado de máquina tem grande sucesso prático em várias aplicações, especialmente em problemas de mineração de dados, reconhecimento de padrões, e problemas de alocação de recursos.



\subsection{k Nearest Neighbor}
Segundo \cite{ranny:2016} o método k-NN é um algoritmo de aprendizado de máquina baseado no aprendizado supervisionado. Ele se baseia na similaridade de distância entre os dados de treinamento na dimensão dos atributos. Ele funciona calculando a distância entre os dados de teste e os dados de treinamento, a fim de determinar a entrada que possui a menor distância e classifica-la.

O k-NN pode ser usado para problemas de classificação e regressão. Em ambos casos as entradas são as mesmas, porém as saídas dependem se ele for usado para classificação ou regressão \cite{altman:1992}:
\begin{itemize}
\item Na classificação, a saída é um membro da classe. Um objeto é classificado pela maioria dos votos entre seus vizinhos, com o objeto sendo designado a classe mais comum entre seus K vizinhos mais próximos. Se K = 1, então ele simplesmente irá pertencer a classe deste único vizinho mais próximo.
\item Na regressão, a saída é o valor da propriedade do objeto. Este valor é o cálculo da média dos valores dos seus K vizinhos mais próximos.
\end{itemize}



\subsection{Redes Neurais Artificiais}
Segundo \cite{Lathika:2014} Redes Neurais Artificiais (RNA) são um paradigma de processamento inspirado pela maneira que o cérebro humano funciona. Uma RNA aprende através de exemplos, e por isto são bastante utilizadas para o reconhecimento de padrões e classificação de dados. O processo de aprendizado em uma RNA envolve o ajuste dos pesos de suas conexões entre cada neurônio artificial. Em um sistema de reconhecimento biométrico as características extraídas são utilizadas como entrada para o treinamento de uma RNA. Após o processo de treinamento for concluído e a RNA convergir, ela pode ser usada para realizar o reconhecimento através de novas entradas biométricas.

Segundo \cite{mitchell:l997}, RNAs são bem aplicadas em problemas onde os dados de treinamento são de sensores muito complexos e possuem ruídos, como imagens e sons obtidos através de câmeras e microfones. E também se aplicam a problemas de decisão, onde representações simbólicas são mais usadas, e consegue atingir resultados comparáveis a arvores de decisão.


\section{Algoritmos de criptografia}
Segundo \cite{schneier:2007} um algoritmo de criptografia, também conhecido como cifrador, é uma função matemática usada para cifragem e decifragem. Para \cite{stallings:2014} algoritmos criptográficos são técnicas para garantir o sigilo e/ou a autenticidade da informação. Os dois ramos principais da criptologia são a criptografia, que é o estudo do projeto dessas técnicas; e a criptoanálise, que trata de frustrar essas técnicas, recuperar informações ou forjar informações que serão aceitas como autênticas.



\subsection{Criptografia Simétrica}
Segundo \cite{stallings:2014} um esquema de encriptação simétrica possui cinco itens que podem ser observados na Figura \ref{fig:modelo-encriptacao-simetrica}:

\begin{itemize}
\item Texto claro: essa é a mensagem ou dados originais, inteligíveis, que servem como entrada do algoritmo
de encriptação.
\item Algoritmo de encriptação: realiza diversas substituições e transformações no texto claro.
\item Chave secreta: também é uma entrada para o algoritmo de encriptação. A chave é um valor independente do texto claro e do algoritmo. O algoritmo produzirá uma saída diferente, dependendo da chave usada no momento. As substituições e transformações exatas realizadas pelo algoritmo dependem da chave.
\item Texto cifrado: essa é a mensagem embaralhada, produzida como saída do algoritmo de encriptação. Ela depende do texto claro e da chave secreta. Para determinada mensagem, duas chaves diferentes produzirão dois textos cifrados distintos. O texto cifrado é um conjunto de dados aparentemente aleatório e, nesse formato, ininteligível.
\item Algoritmo de decriptação: esse é basicamente o algoritmo de encriptação executado de modo inverso.
Ele apanha o texto cifrado e a chave secreta e produz o texto claro original.
\end{itemize}

\begin{figure}[H]
  \centering
  \caption{Modelo simplificado da encriptação simétrica.}
  \includegraphics[width=\textwidth]{Modelo-simplificado-da-encriptacao-simetrica.png}
  \fonte{\cite{stallings:2014}}
  \label{fig:modelo-encriptacao-simetrica}
\end{figure}

Existem diversos tipos de algoritmos de criptografia simétricos e dois exemplos bastante utilizados são: DES e AES.



\subsection{DES - \textit{Data Encryption Standard}}
DES é um algoritmo de chave simétrica de bloco, o tamanho de sua chave é de 56 bits trabalha sobre blocos de 64 bits. Desenvolvido em 1974 e foi o primeiro padrão de criptografia recomendado pelo NIST (\textit{National Institute of Standards and Technology}). Ele pode operar em diferentes modos o que o torna flexível. 
O seu algoritmo começa com uma permutação inicial, realiza dezesseis ciclos de cifragem em bloco e finaliza com outra permutação, este esquema geral para a encriptação DES pode ser observado na Figura \ref{fig:des}. Sua aplicação é bastante popular em diversos domínios como militar e comercial. Existem variações de sua implementação que estendem sua funcionalidade como: 3DES e AES \cite{panda:2016}.


\begin{figure}[H]
  \centering
  \caption{Representação geral do algoritmo de encriptação DES.}
  \includegraphics[width=\textwidth]{DES.png}
  \fonte{\cite{stallings:2014}}
  \label{fig:des}
\end{figure}


\subsection{AES - \textit{Advanced Encryption Standard}}
O AES é um algoritmo de chave simétrica de bloco e foi publicado pelo NIST em 2001 com objetivo de substituir o DES como padrão de criptografia simétrica \cite{stallings:2014}. Diferente dos algoritmos de criptografia de chave pública como RSA, o AES e a maioria dos algoritmos simétricos possuem uma estrutura bastante complexa. Todas as operações realizadas pelo algoritmo são em 8 bits, e envolve operações com números inteiros.

O algoritmo recebe como entrada um bloco de texto claro sem formatação com 128 bits de extensão. O comprimento da chave pode ser de 128, 192 ou 256 bits. E o algoritmo é chamado de AES-128, AES-192 ou AES-256, dependendo do tamanho da chave.  O bloco de entrada é copiado para um \textit{Array} \textbf{Estado}, que é modificado por quatro transformações a cada etapa da encriptação ou decriptação. Os quatro estágios diferentes de transformações utilizados, um de permutação e três de substituição são:
\begin{itemize}
\item \textit{SubBytes}: Utilização de uma S-box para realizar a substituição byte a byte do bloco.
\item \textit{ShiftRows}: Permutação simples.
\item \textit{MixColumns}: Substituição aritmética.
\item \textit{AddRoundKey}: Um XOR bit a bit simples do bloco atual com uma parte da chave expandida.
\end{itemize}

O número de rodadas de transformações depende do tamanho da chave utilizada. São 10 rodadas para RSA-128, 12 rodadas para o RSA-192 e 14 rodadas para o RSA-256. Após a última etapa, o \textbf{Estado} é copiado para a saída, que resulta no texto cifrado. Já o processo de decriptação é praticamente o inverso, porém com a chave diferente.



\subsection{Criptografia Assimétrica}
Segundo \cite{stallings:2014} criptografia assimétrica ou algoritmos de chave pública são baseados em funções matemáticas, em vez de substituição e permutação. Os algoritmos de chave pública envolvem o uso de duas chaves assimétricas, uma pública e uma privada, que são usadas para realizar operações complementares, como encriptação e decripitação ou geração e verificação de assinatura. Os dois algoritmos de chave pública para uso geral mais utilizados são o RSA, e curva elíptica, onde o RSA é o mais adotado.

\subsection{RSA}
Segundo \cite{borodzhieva:2016} o RSA é um dos primeiros algoritmos de chave pública desenvolvidos e ele é fortemente adotado para o uso na transmissão segura de dados. Ele utiliza um par de chaves, sendo uma pública utilizada para encriptação, e uma privada utilizada para decifração. O algoritmo explora o fato de que dois números inteiros \textit{p} e \textit{q} podem ser facilmente multiplicados, mas é muito mais difícil fatorar seu produto \textit{n = p.q}. Então o produto como parte da chave de criptografia pode ser disponibilizado publicamente, enquanto os multiplicadores que são o segredo para a decifração, permanecem privados. Se os números inteiros utilizados no RSA conterem mais de 100 dígitos, a multiplicação pode ser feita em segundos enquanto a fatoração dos multiplicadores (números primos) pode levar bilhões de anos.



\section{SSL, TLS e HTTPS}
Segundo \cite{lien:2011} o \textit{Secure Socket Layer} (SSL) é um protocolo de criptografia, que fornece a comunicação segura sobre a internet, ele faz isto considerando a forma em que é feita a troca de dados entre dois \textit{Sockets}, e sua segurança está na autenticação das partes envolvidas, e na cifragem dos dados transmitidos entre as partes.
Já o \textit{Transport Layer Security} (TLS), estende a segurança do SSL através do uso da técnica chamada de \textit{Handshaking}, que é um processo de negociação onde se estabelece um canal de comunicação seguro entre as duas entidades, antes de começar a comunicação normal.

O SSL oferece serviços básicos de segurança para vários protocolos da camada de transporte, particularmente o \textit{Hypertext Transfer Protocol} (HTTP). O protocolo HTTPS(HTTP over SSL) vem da combinação dos protocolos HTTP e SSL, e ele está presente em todos os navegadores Web modernos. Uma das maneiras de se observar o uso do HTTPS é através do endereço da URL (Uniform Resource Locator) de um \textit{website}, que começa com https:// em vez de http://, e no HTTPS a porta utilizada é a 443, diferente do HTTP que é utiliza a porta 80 por padrão. E em uma comunicação HTTPS os seguintes elementos são encriptados \cite{stallings:2014}:
\begin{itemize}
\item URL do documento solicitado.
\item Conteúdo do documento.
\item Conteúdo dos formulários do navegador.
\item \textit{Cookies} enviados do navegador ao servidor e vice-versa.
\item Conteúdo do cabeçalho HTTP.
\end{itemize}





\section{Tecnologias}
Nesta seção são citadas algumas das tecnologias que foram estudadas para utilização na implementação do sistema seguro de armazenamento de senhas com autenticação biométrica.


\subsection{Scikit-Learn}
\textit{Scikit-learn} é uma biblioteca de aprendizado de máquina de código aberto desenvolvida para a linguagem de programação Python. Ela implementa uma variedade de algoritmos de classificação, regressão e agrupamento, incluindo: \textit{Suport Vector Machines}, \textit{k-Nearest Neighbors}, \textit{Random Forest}, Redes Neurais Artificiais, entre outros. 

Segundo \cite{scikit-learn} a visão do projeto \textit{Scikit-learn} não é fornecer a maior quantidade possível de algoritmos, mas sim fornecer implementações de qualidade destes algoritmos, evitando o conceito de \textit{framework} para que seja simples a integração da biblioteca. Isto é feito utilizando o mínimo possível de objetos diferentes, e dependendo em contêineres de dados em forma de \textit{numpy arrays}. Outro grande foco do projeto é a documentação, contendo mais de 300 páginas incluindo, documentação narrativa, referência de classes, tutoriais, instruções de instalação, e mais de 60 exemplos de aplicação (alguns com problemas práticos do mundo real). A documentação é escrita procurando minimizar o uso de termos específicos de aprendizado de máquina, enquanto mantendo a precisão sobre a descrição dos algoritmos. Algumas das tecnologias que dão base para o \textit{Scikit-learn} são:
\begin{itemize}
\item \textit{Numpy}: A estrutura de dados base usada para dados e modelo de parâmetros. Dados de entrada são apresentados como \textit{numpy arrays}. E também fornecem operações aritméticas base.
\item \textit{Scipy}: Algoritmos eficientes para álgebra linear, representação de matrizes, funções especiais e funções estatísticas básicas.
\item \textit{Cython}: Uma linguagem para combinar C em \textit{Python}. \textit{Cython} torna possível alcançar performance de linguagens compiladas utilizando sintaxe similar a própria do \textit{Python}.
\end{itemize}

Enquanto um dos principais objetivos do \textit{Scikit-learn} é a simplicidade de uso, e é escrita em sua grande parte em linguagem de alto nível, o projeto considera a maximização do uso computacional. Na Tabela \ref{comparacao-scikit-learn}, é possível visualizar a comparação entre o tempo de computação de algum dos principais algoritmos implementados pelas bibliotecas de aprendizado de máquina mais populares em \textit{Python}.

\begin{table}[H]
\centering
\caption{Desempenho de algoritmos de aprendizado de máquina em Python.}
\label{comparacao-scikit-learn}
\begin{tabular}{lcccccc}
\hline
                              & \textit{scikit-learn} & \textit{mlpy} & \textit{pybrain} & \textit{pymvpa} & \textit{mdp} & \textit{shogun} \\ \hline
Support Vector Classification & \textbf{5.2}          & 9.47          & 17.5             & 11.52           & 40.48        & 5.63            \\
Lasso (LARS)                  & \textbf{1.17}         & 105.3         & -                & 37.35           & -            & -               \\
Elastic Net                   & \textbf{0.52}         & 73.7          & -                & 1.44            & -            & -               \\
k-Nearest Neighbors           & 0.57                  & 1.41          & -                & \textbf{0.56}   & 0.58         & 1.36            \\
PCA (9 components)            & \textbf{0.18}         & -             & -                & 8.93            & 0.47         & 0.33            \\
k-Means (9 clusters)          & 1.34                  & 0.79          & *                & -               & 35.75        & \textbf{0.68}   \\
License                       & BSD                   & GPL           & BSD              & BSD             & BSD          & \textit{GPL}    \\ \hline
\multicolumn{2}{l}{- : Não implementado.}             & \multicolumn{5}{l}{⋆ : Não converge em menos de 1 hora.}   
\end{tabular}
\fonte{\cite{scikit-learn}}
\end{table}




\subsection{Scikit-Image}
\textit{Scikit-image} é uma biblioteca de processamento de imagens que implementa algoritmos e utilidades para uso em pesquisas, educação e aplicações industriais. Esta sobre a licença BSD livre para uso e modificação. Ela fornece uma \textit{Application Programming Interface} (API) bem documentada para a linguagem de programação \textit{Python}, e é desenvolvida por uma ativa comunidade internacional de colaboradores \cite{scikit-image}.

Segundo \cite{scikit-image} um dos principais objetivos do projeto é tornar fácil para qualquer usuário começar seu uso rapidamente, especialmente para usuários já familiarizados com outras ferramentas cientificas do \textit{Python}. A \textit{scikit-image} fornece ferramentas robustas para conversão de diferentes tipos de arquivos de imagens, e um grande número de algoritmos que possuem aplicações sobre pesquisas de processamento de imagens. A galeria online de exemplos fornece uma visão geral das funcionalidades disponíveis no pacote de ferramentas, e introduz a maioria dos algoritmos mais utilizados em problemas de processamento de imagens.

O projeto do \textit{scikit-learn} respeita os padrões de estilo de código PEP8, e a documentação de formatação do \textit{NumPy}, a fim de fornecer uma experiência consistente e familiar entre diferentes bibliotecas cientificas desenvoldas em \textit{Python}. A representação dos dados usada são \textit{NumPy arrays} multidimensionais, para garantir esta interoperabilidade entre o ecossistema científico do \textit{Python}. A maior parte da API do \textit{scikit-image} é desenvolvida como uma interface funcional, o que permite que uma função pode ser simplesmente ser aplicada ao resultado de outra. O código fonte é em sua maioria escrito em \textit{Python}, porém algumas partes onde performance é crítica são implementadas em \textit{Cython}, que é um compilador de otimização para \textit{Python} \cite{scikit-image}.


\subsection{OpenCV}
O OpenCV (Open Source Computer Vision Library) foi originalmente desenvolvido pela Intel em 2000, e é uma biblioteca multiplataforma (disponível nas linguagens de programação C/C++, Java, Python e Visual Basic) de código livre para o uso acadêmico e comercial. O \textit{OpenCV} é dividido em uma estrutura modular, o que significa que o o pacote inteiro contém várias bibliotecas estáticas ou compartilhadas. Os seguintes módulos estão disponíveis:

\begin{itemize}
\item Funcionalidades principais: Um modulo compacto que defini as estruturas de dados básicas e funções usadas por todos os outros módulos.
\item Processamento de Imagem: modulo que contém filtros de imagens lineares e não lineares, funções de transformação geométrica, conversão do espaço de cores, histogramas, entre outros.
\item Video: Módulo para análise de vídeos que inclui funções para estimativa de movimento, remoção de fundo, e algoritmos para detecção de objetos.
\item Calib3d: Algoritmos básicos de geometria de visão múltipla, calibração de câmera, estimativa de pose de objetos, e elementos de reconstrução 3D.
\item Features2d: Detecção de características, descritores, e descritores de comparação.
\item Objdetect: Detecção de objetos e instancias de classes predefinidas, como: faces, olhos, carros, entre outros.
\item Highgui: Interface simples e fácil de ser utilizada.
\item Video I/O: Interface para captura de vídeos e de \textit{codecs}.
\item GPU: Algoritmos acelerados por Unidades de Processamento Gráfico (GPU) de diferentes módulos do \textit{OpenCV}.
\end{itemize}

Além de outros módulos auxiliares, como FLANN, \textit{Google Test Wrapers}, \textit{Python bindings}, entre outros.



\section{Bases de dados biométricos}
Existem uma variedade de bases de dados de características biométricas, principalmente de faces e voz, que é o objetivo deste trabalho. O trabalho de \cite{violato:2013} desenvolveu uma base de dados biométrica com amostras de face e voz de indivíduos brasileiros. Esta base de dados contempla capturas de vídeos, com áudio de pessoas de diferentes idades e gêneros, obtidos em diferentes cenários de uso e por meio de três canais distintos: \textit{smartphones}, \textit{notebooks} e chamadas telefônicas. 

Na própria documentação do \textit{OpenCV} \cite{opencv}, é citado a base de imagens de faces ORL e YALE. Sendo a segunda dividida em duas partes. A \textit{YALE Facedatabase A} contendo 165 imagens em tons de cinza no formato GIF (acrônimo para Graphics Interchange Format) de 15 pessoas, são 11 imagens para cada uma, e cada uma com expressões faciais e configurações diferentes. Já a \textit{YALE Facedatabase B} estendida contém 16128 imagens de 28 pessoas, sobre 9 diferentes poses e 64 condições de iluminação diferentes \cite{yale:2001}. Estas bases são de uso livre para propósito de pesquisa.
\chapter{Trabalhos relacionados}
\label{trabalhos-relacionados}

A partir da pesquisa em bases de dados científicas foram selecionados alguns trabalhos relacionados aos temas de autenticação biométrica e sistemas de gerenciamento de senhas. Cada um destes trabalhos é descrito nas seções seguintes.

\section{Multimodal Biometrics for Enhanced Mobile Device Security}
Em seu trabalho \cite{gofman:2016} desenvolveu um sistema de autenticação biométrica multi modal para dispositivos móveis, onde são utilizados reconhecimento facial e reconhecimento de voz para o processo de autenticação, a fim de tornar mais robusto e seguro o uso de biometria em dispositivos móveis. Uma grande parte dos dispositivos móveis do mercado já suportam reconhecimento de faces, voz e digitais. E para a parte de reconhecimento facial do sistema foi utilizado a técnica  \textit{FisherFaces}, pois ela funciona bem em casos de imagens capturadas em condições variadas, o que é o caso de imagens de faces obtidas pelos dispositivos móveis. Já para o Reconhecimento de voz foram utilizados duas técnicas: \textit{Hidden Markov Models} (HMM) baseada na \textit{Mel-Frequency Cepstral Coefficients} (MFCCs) como caracteristicas de voz, que foi a base para esquema de combinação de resultados baseado em nota, e \textit{Linear Discriminant Analysis} (LDA) que foi a base para o esquema de combinação de resultados baseado em características. Ambas as técnicas são capazes de reconhecer a voz do usuário independente das frases faladas. A qualidade da imagem facial é calculada baseando-se na luminosidade, nitidez e contraste, enquanto a qualidade da gravação de voz é baseada na relação sinal-ruído. O treinamento e teste do sistema é feito com vídeos de pessoas segurando a câmera do dispositivo em frente ao seu rosto enquanto falam uma certa frase. Para cada vídeo a face é detectada utilizando o algoritmo de Viola-Jones, e a gravação de som é processada removendo todas frequências fora do nível da voz humana (85Hz-255Hz). O treinamento do sistema foi feito utilizando vídeos de metade das pessoas da base de dados (27 de um total de 54), enquanto todos os vídeos foram utilizados nos testes. A maioria dos vídeos foram coletados em condições controladas com boa luz e pouco ruído de fundo, e também foram adicionadas algumas amostras de vídeo e som de baixa qualidade para aumentar a chance do algoritmo identificar corretamente o usuário em condições similares. O melhor resultado foi obtido no cálculo de combinação baseado em características, que obteve as taxas de erro de 4,3\% no reconhecimento facial, 34,7\% no reconhecimento de voz e 2,1\% 
na combinação. O sistema baseado em faces e voz foi implementado em um dispositivo \textit{Samsung Galaxy S5}, obteve grande precisão na autenticação, quando comparado a sistemas onde são utilizados somente faces ou voz. Futuramente planeja-se explorar outras técnicas de reconhecimento facial, como \textit{Gabor wavelets} e \textit{Histogram Oriented Gradients} (HOG).



\section{Facial Recognition using Histogram of Gradients and Support Vector Machines}
O trabalho apresentado por \cite{julina:2017}, tem como objetivo principal realizar o reconhecimento facial utilizando as técnicas de \textit{Suport Vector Machines} (SVM) e histograma de gradientes. O sistema desenvolvido foi projetado para lidar com problemas de pose e iluminação no reconhecimento de faces. O sistema conta com a base de dados de faces AT \& T, que contém faces de 40 pessoas, com 10 amostras de cada. Todas as imagens são do mesmo tamanho, e contem diferentes poses e expressões faciais da mesma pessoa, com iluminação constante. São acrescentados a base mais um conjunto de 10 imagens de uma pessoa em diferentes poses, tiradas em condições de luz diferentes. E esta base é dividida em dois conjuntos, o de treinamento e o de teste. Para que se obtenha boas taxas de reconhecimento é preciso realizar o pré-processamento de novas imagens antes de adiciona-las a base de dados. Todas as novas imagens são recortadas utilizando o algoritmo de Viola-Jones, e redimensionadas para o padrão em comum de 112 X 92 pixels. A extração de características e classificação das imagens é obtida pelo histograma de gradientes. Dentre as imagens das 41 pessoas na base de dados totalizando 369 imagens, são separadas 1 de cada conjunto e utilizadas para a base de testes totalizando 41 imagens. Estas bases são utilizadas para o treinamento do algoritmo SVM, e quando uma das imagens da base de testes é utilizada na entrada do algoritmo ele retorna todas as imagens reconhecidas na base de treinamento. Para validação do modelo de classificação foi formulado uma matriz de confusão sobre os dados de teste. Para isto são separados os acertos e os erros, cada um em corretamente e incorretamente identificados, e o nível de precisão é calculado através da soma entre os identificados corretamente (37) e os identificados incorretamente (0), dividido pelo total da base de testes (41), que resultou em 90.2439\% de precisão. O método proposto é capaz de realizar o reconhecimento de faces em diferentes poses e condições de iluminação, demonstrando baixa quantidade de falsos positivos e uma melhoria na precisão de detecção. Para trabalhos futuros busca-se o reconhecimento de faces em imagens 3D, utilizando os conceitos de \textit{deep learning} onde redes neurais convolucionais são utilizadas.



\section{Voice Recognition using k Nearest Neighbor and Double Distance Method}
O trabalho apresentado por \cite{ranny:2016}, tem como objetivo principal realizar reconhecimento de voz utilizando as técnicas de \textit{k Nearest Neighbor}  (k-NN) e \textit{Double Distance Method}. Reconhecimento de voz é um dos sistemas que se desenvolveu baseado em reconhecimento de padrões. O processo de reconhecimento de voz é baseado em reconhecimento de padrões e pode ser desenvolvido utilizando vários tipos de métodos, entre eles estão, \textit{Dynamic Time Wrapping} (DTW), \textit{Linear Vector Quantization} (LVQ), Redes Neurais artificiais (RNA) entre outros. A estrutura de um sistema de reconhecimento de voz consiste de duas etapas, que são o treinamento e teste. Na etapa de treinamento é feita a extração da característica da voz utilizando \textit{Mel Frequency Cepstrum Coefficients} (MFCC). O objetivo do MFCC é converter o sinal do domínio de tempo para o domínio de frequência. Sinais no domínio do tempo são mais difíceis de serem processados e analisados por sua alta quantidade e complexidade dos dados. Já o sinal no domínio da frequência é simples de ser analisado pois o padrão do sinal é obtido dos dados. O próximo passo é o processo de teste, onde é utilizado o método kNN com o método de distância dupla proposto e k = 1. Sistemas de reconhecimento de voz precisam de uma grande quantidade de dados de treinamento em ordem para aumentar seu nível de precisão. Normalmente, o algoritmo 1-NN calcula a média dos dados de treinamento e usa este valor para representar uma classe. O resultado do reconhecimento é obtido através do cálculo da menor distância entre o teste e a média da classe. Isto faz com que o processo de reconhecimento leve mais tempo. E o sistema de reconhecimento de voz também necessita de um método para cuidar dos dados aberrantes para aumentar a precisão, pois o 1-NN não é adequado para este tipo de aplicação. Foram feitos dois experimentos para comprovar a melhoria da precisão no reconhecimento de voz. No primeiro experimento é utilizado o método 1-NN e são calculados utilizando o cálculo de média, e baseado neste teste o nível de precisão é de 84,85\%. O segundo experimento é realizado utilizando os mesmos dados do primeiro e o método de distância dupla, e o resultado dos testes demonstra um nível de precisão de 96,97\%. O método de distância dupla proposto demonstrou melhoria em sua utilização com o método 1-NN, especialmente em amostras com \textit{outliers}. A comparação entre o método de distância dupla e outros métodos de aprendizado de máquina como \textit{Hidden Markov Model}, Redes Neurais e \textit{Linear Predictive Code} podem ser tópicos de pesquisas futuras.



\section{Artificial Neural Network Based Multimodal Biometrics Recognition System}
O trabalho apresentado por \cite{Lathika:2014}, tem como objetivo principal o desenvolvimento de um sistema de reconhecimento biométrico multimodal para prover segurança e autenticação, utilizando três características biométricas que são: face, orelha e maneira de andar. O projeto do sistema proposto consiste de seis módulos, que são: aquisição da imagem, extração de características, treinamento da rede neural artificial, normalização, combinação e decisão final. A primeira etapa no desenvolvimento de um sistema de reconhecimento biométrico é a obtenção dos dados biométricos do sensor de \textit{hardware}. E o resultado é uma imagem ou sinal capturado da característica biométrica. O módulo de extração de características é responsável pesa transformação dos dados de entrada em conjuntos de características, e para este trabalho é utilizado a técnica de \textit{Discrete Wavelet Transform} (DWT). Esta técnica decompõem os sinais de entrada em conjuntos de funções básicas que são chamadas de \textit{wavelets}. Para realizar a classificação da biometria é utilizado uma rede neural artificial do tipo \textit{feedforward}. Uma rede neural artificial (RNA) é um paradigma de processamento de informações que é inspirado pela maneira que o cérebro humano funciona. O módulo de normalização tem a função transformar os dados em um único domínio para que seja possível sua comparação e classificação. Para este sistema o método de normalização utilizado é o \textit{z-score}. Para realizar o processo de combinação das notas calculadas para as três características biométricas, é utilizado a técnica de soma com pesos. Cada característica é processada separadamente e são atribuídas notas para cada entrada. Então a nota composta é calculada dependendo da precisão de cada biometria. Este tipo de combinação indica a proximidade entre os vetores de características a serem identificados. Para a simulação do sistema foram utilizadas várias bases de dados das biometrias propostas, que foram: base de dados de maneira de caminhar CASIA e GAID; USTB base de dados de orelhas; AR e UWA para faces e orelhas; e ORL base de faces. As imagens foram pré-processadas utilizando o filtro de \textit{Weiner}, a fim de amenizar ruídos e borrões das imagens. Após as imagens passam pelo processo de ampliação de contraste utilizando as técnicas de \textit{imadjust}, \textit{histeq} e \textit{adapthisteq}. São extraídas as características das imagens utilizando a técnica de DWT, e os vetores de características são utilizados como entrada para a RNA e é iniciado o processo de treinamento. Os testes são realizados utilizando 10, 20, 30 e 40 amostras e obteve um nível de precisão na identificação das características biométricas de 99.21\% com o maior número de amostras. O sistema obteve excelentes resultados de precisão no reconhecimento com alta performance, e se demonstrou superior a sistemas unimodais sobre bases de dados de imagens de faces, orelhas e maneira de caminhar. Como melhorias futuras, o sistema pode considerar o fator de envelhecimento das pessoas, porém este não é muito prático pois requer a extração de leituras biométricas sobre um período de tempo de meses ou anos.



\section{Sesame: A Secure and Convenient Mobile Solution for Passwords}
O trabalho apresentado por \cite{aliasgari:2015}, tem como objetivo principal desenvolver um sistema de de gerenciamento de senhas seguro para dispositivos móveis, utilizando reconhecimento de voz e de fala para realizar a autenticação e criptografia para armazenar de forma segura as senhas. O desenvolvimento do sistema se justifica pois os \textit{smartphones} atuais são capazes de capturar processar e armazenar informações pessoais facilmente, e a cada dia o número de dispositivos está aumentando. O Sesame realiza o armazenamento das senhas criptografadas, e para cada senha são utilizadas diferentes chaves para o algoritmo de criptografia, estas chaves são criptografadas com uma chave que é derivada da senha mestre. O armazenamento das senhas já criptografadas é feito no dispositivo e na nuvem de preferência do usuário. Já as chaves utilizadas para criptografia de cada senha são armazenadas em servidores do Sesame. O processo de autenticação é realizado através do uso da senha mestre que o usuário deve informar na primeira vez que iniciar o aplicativo, ou através do reconhecimento de voz. O Sesame coleta uma amostra de 10 segundos da voz do usuário em sua primeira instalação para fazer a calibragem do reconhecimento de voz. Esta amostra é processada no servidor e é gerado um \textit{Gaussian Mixture Model} (GMM). O servidor gera uma ID única e retorna para a o dispositivo móvel, enquanto no dispositivo já são geradas um par de chaves pública e privada, e uma chave de criptografia de 256bit. Para o usuário visualizar suas senhas salvas ele fornece uma amostra de voz que é enviada para o servidor do Sesame, onde é aplicado o reconhecimento de voz, se a amostra estiver dentro de um limite aceitável o usuário que deseja acessar o sistema será autenticado. Em ordem de definir o limite aceitável para a autenticação do usuário foram coletadas amostras de voz de 110 pessoas e conduzidos 23.409 testes de verificação. A técnica de \textit{Alize/LIA RAL toolkit} é utilizada para o reconhecimento de voz funciona independente das frase de amostra, e  para o reconhecimento de fala é utilizada a técnica \textit{Sphinx toolkit}. Ambas são de código aberto e foram desenvolvidas por universidades americanas. O trabalho proposto resultou no desenvolvimento de uma aplicação conveniente e segura para o armazenamento de senhas e dados privados de usuários. Para trabalhos futuros propõe-se a integração de outras modalidades de biometria, e a generalização do sistema para que possa realizar o armazenamento seguro de qualquer tipo de informação.


\section{Multimodal biometric authentication based on voice, face and íris}
O trabalho apresentado por \cite{barbu:2015}, tem como objetivo principal o desenvolvimento de um sistema de autenticação biométrica multimodal baseado em voz, face e íris. Para o sistema proposto são utilizadas três identificadores biométricos: voz, face e íris. Os dados destas características são fáceis de serem coletados, pois podem ser capturadas através de simples microfones e câmeras. O método de reconhecimento de voz utilizado é independente do texto da amostra, são extraídas as características do sinal através da técnica de \textit{cepstral melodic analysis}. Após as frequências são convertidas para escala de mel que é mais apropriada para voz então é calculado um \textit{delta mel frequency cepstral coefficients}
(DDMFCC) para cada amostra, que então podem ser utilizados como entrada para um algoritmo de classificação supervisionada como: mínima distância média ou k-NN. Para o reconhecimento facial é utilizado a técnica \textit{Scale Invariant Feature Transform} (SIFT), publicada por David Lowe em 1999. Esta técnica faz a extração de pontos chave de uma imagem em ordem de produzir uma descrição. Estas características não são afetadas pelo tamanho da imagem, sua orientação ou mudanças de iluminação, o que a torna robusta para descrição de faces. A técnica de SIFT resulta em um vetor de características, onde para que duas faces sejam consideradas da mesma pessoa ambos vetores precisam estar em uma distância próxima um do outro.
Para a tarefa de reconhecimento das características faciais extraídas também são utilizadas as técnicas de mínima distância média ou \textit{K-Nearest-Neighbour} (K-NN). E finalmente para o reconhecimento de íris, é proposta uma técnica que explora a sua distribuição de cores. Uma imagem da íris é utilizada para o reconhecimento, e são consideradas somente a parte colorida da imagem, descartando a pupila no centro da imagem. Esta imagem é então dividida em quatro setores e para cada um é calculado seu histograma, e então o resultado deste processamento é obtido concatenando os quatro histogramas em um vetor de cores. Para a identificação da íris a classificação de mínima distância sobre o vetor de cores pode ser usada.
Com todas as três características biométricas processadas o sistema realiza o processo de combinação. Este processo é realizado pode obter melhores resultados com as entradas das características originais de biometria, porém a combinação neste nível é desafiante pois os dados são incompatíveis em sua primeira forma, por isto no sistema proposto foi considerado a combinação no módulo de decisão. Foram explorados diferentes estratégias de combinação que são: \textit{majority voting}, \textit{behavior knowledge space method}, \textit{AND/OR rules and weighted based on Dempster-Shafter theory of evidence}. O escolhido para o sistema foi o de \textit{majority voting} aplicado sobre as três características. Foram realizados testes utilizando o sistema desenvolvido, e cada um dos três métodos de reconhecimento obteve resultados equivalentes ao estado da arte, e o sistema desenvolvido obteve uma taxa de reconhecimento de aproximadamente 90\%. O sistema desenvolvido se demonstrou eficiente para a autenticação de usuários, e o sistema pode ser melhorado em trabalhos futuros integrando outro reconhecimento biométrico como o de digitais.



\section{Quadro Comparativo dos trabalhos relacionados}
A Tabela \ref{comparativo} mostra um comparativo entre os trabalhos relacionados analisados, levando em consideração as características biométricas utilizadas, as técnicas de reconhecimento biométrico.

\begin{table}[H]
\centering
\caption{Comparativo entre os trabalhos relacionados analisados}
\label{comparativo}
\begin{tabular}{|p{4cm}|p{4cm}|p{7cm}|}
\hline Trabalho Relacionado &            Biometria           &    Técnicas \\ 
\hline \cite{gofman:2016}   & Face, Voz.                    & Fisherfaces, \textit{Hidden Markov Models}, \textit{Linear Discriminant Analysis}.\\ 

\hline \cite{julina:2017}   & Face.                         & \textit{Suport Vector Machines}, Histograma de gradientes.\\ 
\hline \cite{ranny:2016}    & Voz.                           & \textit{k Nearest Neighbor}, \textit{Double Distance Method}.\\ 
\hline \cite{Lathika:2014}  & Face, Orelha, Modo de Andar. & \textit{Discrete Wavelet Transform}, Rede Neural Artificial.\\ 
\hline \cite{aliasgari:2015}& Voz, Fala.                     & \textit{Gaussian Mixture Model}, \textit{Alize/LIA RAL toolkit}, \textit{Sphinx toolkit}.\\ 
\hline \cite{barbu:2015}    & Voz, Face, Íris               &  Miníma Distância Média, \textit{k Nearest Neighbor}, \textit{Scale Invariant Feature Transform}.\\ 
\hline
\end{tabular}
\fonte{(AUTORES, 2017)}
\end{table}

Na tabela \ref{comparativo}, a coluna "Biometria" refere-se as características biométricas utilizadas no trabalho. Todos os trabalhos realizaram a tarefa de reconhecimento de uma ou mais biometria, e os trabalhos \cite{gofman:2016}, \cite{Lathika:2014}, \cite{aliasgari:2015} e \cite{barbu:2015} aplicaram este reconhecimento em um sistema de autenticação, enquanto os outros tem como objetivo apenas o reconhecimento. Também é possível perceber que os trabalhos possuem características biométricas em comum que são: faces e voz.

A coluna "Técnicas" refere-se as técnicas e algoritmos aplicados para realizar o reconhecimento biométrico. É possível perceber que todos os trabalhos utilizam algoritmos de aprendizado de máquina. E alguns destes algoritmos são utilizados em mais de um trabalho e até mais de uma característica biométrica, como é o caso do \textit{K-Nearest Neighbor} (KNN), que é utilizado para reconhecimento tanto de faces quando de voz.

Entre os seis trabalhos apresentados o que mais se aproxima do objetivo deste trabalho foi o de \cite{aliasgari:2015}, que tem como objetivo principal desenvolver um sistema de gerenciamento de senhas seguro para dispositivos móveis, utilizando reconhecimento de voz e de fala para realizar a autenticação e criptografia para armazenar de forma segura as senhas. A única diferença é que o processamento das biometrias é feito pelos servidores e não diretamente no dispositivo móvel.

Os trabalhos relacionados serviram de base para o desenvolvimento deste trabalho de conclusão de curso. Buscou-se analisar trabalhos que tivessem o objetivo similar, porém que utilizaram diferentes biometrias e técnicas de reconhecimento, a fim de definir as melhores combinações para o desenvolvimento do sistema proposto neste trabalho.
\chapter{Metodologia}
\label{metodologia}

Este capítulo apresenta a caracterização da pesquisa e os procedimentos metodológicos utilizados para o alcance dos objetivos. A pesquisa é exploratória, a partir da análise dos trabalhos relacionados sobre os temas de autenticação biométrica e sistemas de gerenciamento de senhas. O objetivo é procurar diferentes técnicas para o reconhecimento facial e de voz, a fim de aplicar as que possuem os melhores resultados no sistema de gerenciamento de senhas desenvolvido. 

Sobre o levantamento bibliográfico são pesquisados trabalhos relacionados a aplicação de técnicas de aprendizado de máquina no reconhecimento de características biométricas para autenticação em gerenciadores de senhas, e técnicas de criptografia. O ambiente de pesquisa é bibliográfico, onde serão buscadas fontes através de livros sobre segurança e criptografia de dados, autenticação utilizando características biométricas, aprendizado de máquina, artigos, outros trabalhos de conclusão desenvolvidos, Internet.

O procedimento de validação do sistema será qualitativo, onde será feita uma síntese dos resultados obtidos, evidenciando as contribuições, relatando as limitações do estudo, relacionando os fatos verificados com a teoria. Além disso, serão feitos testes sobre o processo de autenticação através do uso de biometria e algoritmos de aprendizado de máquina.

\section{Bibliometria quantitativa}

A bibliometria foi realizada a partir das bases de dados  \textit{IEEE Xplore} e \textit{ACM Digital Library}. As pesquisas nas bases foram feitas tanto com as sentenças completas como \textit{Biometric Authentication} quanto com as palavras separadas como \textit{Biometric AND Authentication}, sendo que todas as pesquisas foram realizadas com as expressões em inglês.

Apesar de existirem várias publicações relacionadas ao tema de autenticação biométrica, percebe-se que há um número bem menor de trabalhos relacionados aos temas abordados neste projeto. A análise considerou apensa publicações a partir de 2010. As Tabelas \ref{bibliometria separada} e \ref{bibliometria completa} mostram, respectivamente, os valores obtidos a partir das pesquisas utilizando as palavras separadas e sentenças completas. As células marcadas com traços nas tabelas são pesquisas já realizadas de acordo com a orientação "linha e coluna".

\begin{table}[H]
\centering
\caption{Bibliometria realizada com as palavras separadas}
\label{bibliometria separada}
\begin{tabular}{|l|c|c|c|c|}
\hline
\multicolumn{1}{|c|}{{\textit{\begin{tabular}[c]{@{}c@{}}IEEE Xplore e \\ ACM Digital Library \\ – Critério de pesquisa:\\ “Linha” and “Coluna”\end{tabular}}}} & \multicolumn{2}{c|}{\textit{Biometric Authentication}} & \multicolumn{2}{c|}{\textit{Password Management}} \\ \cline{2-5} 
\multicolumn{1}{|c|}{} & \textit{\begin{tabular}[c]{@{}c@{}}IEEE\\ Xplore\end{tabular}} & \textit{\begin{tabular}[c]{@{}c@{}}ACM\\ Digital Library\end{tabular}} & \textit{\begin{tabular}[c]{@{}c@{}}IEEE\\ Xplore\end{tabular}} & \textit{\begin{tabular}[c]{@{}c@{}}ACM\\ Digital Library\end{tabular}} \\ \hline
\textit{Biometric Authentication} & 2435 & 290 & - & - \\ \hline
\textit{Password Management} & 59 & 16  & 350 & 144 \\ \hline
\end{tabular}
\fonte{(AUTORES, 2017)}
\end{table}


\begin{table}[H]
\centering
\caption{Bibliometria realizada com a sentença completa}
\label{bibliometria completa}
\begin{tabular}{|c|c|c|c|c|}
\hline
{\textit{\begin{tabular}[c]{@{}c@{}}IEEE Xplore e \\ ACM Digital Library \\ – Critério de pesquisa:\\ “Linha” and “Coluna”\end{tabular}}} & \multicolumn{2}{c|}{\textit{Biometric Authentication}}            & \multicolumn{2}{c|}{\textit{Password Management}}                                                                \\ \cline{2-5} 
& \textit{\begin{tabular}[c]{@{}c@{}}IEEE\\ Xplore\end{tabular}} & \textit{\begin{tabular}[c]{@{}c@{}}ACM\\ Digital Library\end{tabular}} & \textit{\begin{tabular}[c]{@{}c@{}}IEEE\\ Xplore\end{tabular}} & \textit{\begin{tabular}[c]{@{}c@{}}ACM \\ Digital Library\end{tabular}} \\ \hline
\textit{Biometric Authentication} & 2435 & 2870 & - & - \\ \hline
\textit{Password Management} & 59 & 390 & 350 & 52444 \\ \hline
\end{tabular}
\fonte{(AUTORES, 2017)}
\end{table}



\section{Procedimentos Metodológicos}
Inicialmente, foram pesquisados trabalhos relacionados para identificar o estado da arte sobre o processo de autenticação biométrica. Sobre os trabalhos relacionados estudadas foram selecionados os que utilizaram características biométricas e comportamentais que pudessem ser coletadas através de um \textit{Smartphone} de forma conveniente. Foi feita uma análise das biometrias utilizadas nos trabalhos e os resultados obtidos em relação ao objetivo de autenticação biométrica.

Após o estudo e escolha das biometrias utilizadas no sistema, foi feito uma comparação sobre as diferentes técnicas e algoritmos utilizados para realizar o reconhecimento das mesmas. A partir dos métodos que tiveram melhores taxas de reconhecimento, foram pesquisadas tecnologias de código livre possam ser utilizadas para o desenvolvimento do sistema. Como a maioria dos algoritmos de reconhecimento biométrico vem da área de aprendizado de máquina, foram estudados bibliotecas como: \textit{Scikit-Learn}, \textit{Scikit-Image} e \textit{OpenCV}.

Também foram pesquisadas bases de características biométricas para o treinamento dos algoritmos de aprendizado de máquina utilizados no sistema, e que também possam ser utilizadas como bases de teste para validação do processo de autenticação biométrica no sistema de gerenciamento de senhas. Já para a parte de segurança do sistema foram estudadas as principais técnicas de criptografia utilizadas atualmente para o armazenamento seguro de dados. Na figura \ref{fig:procedimentos-metodologicos}, é possível visualizar um resumo dos procedimentos metodológicos definidos para o desenvolvimento do trabalho.

\begin{figure}[H]
  \centering
  \caption{Procedimento metodológicos.}
  \includegraphics[width=\textwidth]{images/procedimentos-metodologicos.png}
  \fonte{(AUTORES, 2017)}
  \label{fig:procedimentos-metodologicos}
\end{figure}

\chapter{Modelagem do sistema}
\label{modelagem}

Para o desenvolvimento do sistema de gerenciamento de senhas seguro foram estudados diferentes algoritmos de reconhecimento de características biométricas, analisando suas vantagens na aplicação do reconhecimento de faces. Com este conhecimento foi avaliado quais destas técnicas apresentam melhores resultados, em relação a eficiência e taxa de reconhecimento. Também foram estudados diferentes algoritmos de criptografia para o armazenamento das senhas de forma segura no sistema. As principais etapas do funcionamento do sistema são apresentadas na Figura \ref{fig:visao-geral}.

\begin{figure}[H]
  \centering
  \caption{Visão geral do sistema de gerenciamento de senhas.}
  \includegraphics[width=\textwidth]{Imagem.png}
  \fonte{(AUTORES, 2018)}
  \label{fig:visao-geral}
\end{figure}

O sistema foi dividido em dois módulos, um responsável pela autenticação biométrica, e o outro responsável pela criptografia das senhas. O módulo de autenticação biométrica é responsável por realizar os processos de cadastro, calibragem, e autenticação de usuários, e para isto ele faz o uso das técnicas de reconhecimento biométrico. O módulo de criptografia de senhas tem como objetivo principal adicionar uma camada extra de segurança sobre o uso de biometria, para que mesmo que os dados sejam obtidos sem a autenticação do usuário, eles não sejam facilmente decifrados. Este módulo é responsável a criptografia dos dados, e também a transformação da senha cifrada em texto claro para o usuário autenticado visualiza-la. Quando o usuário estiver autenticado o sistema abre uma sessão onde estão disponíveis todas informações do usuário, incluindo a chave de criptografia.

O fluxo principal do sistema acontece de forma simples, primeiro são coletadas as leituras das características biométricas do usuário através do uso de câmera. São coletadas várias imagens da face do usuário, preferencialmente em diferentes expressões faciais. Estas informações são processadas através do uso de algoritmos de pré-processamento, reconhecimento e classificação, onde a combinação dos resultados defini se a autenticação é válida ou não. Sendo autenticado, o usuário tem acesso ao sistema que permite realizar a leitura, cadastro e alteração de informações de \textit{login} como usuário e/ou e-mail, descrição (nome ou URL do serviço), e senha.

As senhas cadastradas no sistema são cifradas utilizando uma chave onde uma parte dela é definida pelo usuário e a outra parte é controlada pelo sistema. O sistema utiliza um algoritmo de criptografia simétrica responsável pela cifragem e decifragem das senhas e outras informações, antes de serem armazenadas no banco de dados de forma segura.
O sistema também contará com funções para gerar senhas automaticamente, baseando-se em boas práticas de segurança e definição da chave de cifragem. A aplicação desenvolvida é para a plataforma web, e os dados serão armazenados em um banco de dados relacional.


\section{Funcionalidades do Sistema} 
O sistema visa realizar o armazenamento das senhas de forma segura e conveniente para os usuários, e contará com as seguintes funcionalidades:
\begin{itemize}
\item Armazenamento de senhas e informações adicionais criptografadas.
\item Geração automática de senhas complexas.
\item Autenticação biométrica através do reconhecimento facial.
\end{itemize}


\section{Requisitos Funcionais}
A Tabela \ref{requisitos-funcionais} representa a lista dos requisitos funcionais do projeto. Estes representam as funções que o sistema deve ser capaz de realizar em termos de tarefas e serviços.

\begin{table}[H]
\centering
\caption{Requisitos Funcionais.}
\label{requisitos-funcionais}
\begin{tabular}{|p{1cm}|p{14cm}|}
\hline
1.1 & O sistema deve permitir a inclusão, alteração, visualização e remoção de senhas.         \\ \hline
1.2 & O sistema deve permitir adicionar informações adicionais para cada senha cadastrada.     \\ \hline
1.3 & O sistema não deve limitar a quantidade de senhas cadastradas.                           \\ \hline
1.4 & O sistema deve realizar o cadastro biométrico em seu primeiro uso.                       \\ \hline
1.5 & O sistema deve possibilitar a escolha da chave utilizada na criptografia.                \\ \hline
1.6 & O sistema deve possuir uma forma alternativa à autenticação biométrica.                  \\ \hline
1.7 & O sistema deve possibilitar a autenticação biométrica através de faces.            \\ \hline
1.8 & O sistema deve possibilitar a autenticação por meio da senha mestre.                     \\ \hline
1.9 & O sistema deve ser disponível em computadores e \textit{smartphones} com acesso a internet.\\ \hline
1.10 & O sistema deve gerar senhas com diferentes níveis de complexidade.\\ \hline
\end{tabular}
\end{table}


\section{Requisitos Não-funcionais.}
A Tabela \ref{requisitos-nao-funcionais} representa a lista dos requisitos não funcionais do projeto. Estes representam os requisitos relacionados aos termos de desempenho, usabilidade, confiabilidade e segurança.

\begin{table}[H]
\centering
\caption{Requisitos Não-funcionais}
\label{requisitos-nao-funcionais}
\begin{tabular}{|p{1cm}|p{14cm}|}
\hline
1.1 & O sistema deve permitir visualização de senhas somente com a autenticação do usuário.          \\ \hline
1.2 & O sistema deve armazenar as senhas de forma segura utilizando criptografia.                    \\ \hline
1.3 & O sistema deve realizar a autenticação biométrica de forma rápida.                             \\ \hline
1.4 & O reconhecimento facial deve ser capaz de reconhecer faces em ambientes com pouca luminosidade.\\ \hline
1.5 & O sistema deve ser desenvolvido para a plataforma Web.                                     \\ \hline
1.6 & O sistema deve estar disponível 24/7.                                                          \\ \hline
1.7 & Somente o usuário deve ser capaz de acessar seus dados.                                        \\ \hline
1.8 & O sistema deve encerrar a sessão do usuário automaticamente após 5 minutos de inatividade.     \\ \hline
1.9 & O sistema deve bloquear o acesso de usuários não legítimos.     \\ \hline
1.10 & O sistema deve bloquear temporariamente após múltiplas tentativas de login sem sucesso.     \\ \hline
\end{tabular}
\end{table}


\section{Diagrama de classes}
A Figura \ref{fig:diagrama-de-classes} representa o diagrama de classes definido para representar a estrutura e relação das classes que irão servir de modelo para os objetos no sistema. O diagrama contém 6 classes, que são: Usuario, Senha, Arquivo, Sessao, Criptografia e ReconhecimentoBiometrico. A classe Usuario é a classe principal, ela representa as informações relevantes ao usuário do sistema, que são: nome do usuário, senha mestre, e-mail, chave utilizada na criptografia, e outras informações relevantes ao processo de autenticação biométrica. A classe Arquivo representa os arquivos que serão obtidos através das leituras biométricas como imagens da face do usuário, e gravações de voz. Cada usuário terá um conjunto de arquivos que serão coletados durante seu primeiro contato com a aplicação. Estes arquivos passarão pelo processo de pré-processamento antes de serem armazenados e utilizados para treinamento dos algoritmos de reconhecimento.

Já a classe Senha é o modelo da principal funcionalidade do sistema, que é o armazenamento de senhas. cada usuário irá poder ter uma quantidade ilimitada de senhas cadastradas (dependendo do espaço de armazenamento disponível no dispositivo), e senhas criptografadas com a chave definida pelo usuário antes de serem armazenadas. Para cada senha cadastrada pode se ter também informações adicionais como a URL e descrição do serviço.

As três classes: Sessao, Criptografia e ReconhecimentoBiometrico, são classes responsáveis pelos processos do sistema. A Sessão é responsável por controlar o acesso e navegação no sistema, e ela possui as informações do usuário autenticado, como usuário, senha, chave criptografada da sessão e o horário que a sessão foi criada. A classe Criptografia tem duas funções que são a cifragem e decifragem das senhas utilizando a chave definida pelo usuário.
A classe ReconhecimentoBiometrico é responsável por aplicar as técnicas de reconhecimento estudadas, e ela faz o processamento das biometrias tanto no cadastro de um novo usuário quanto na autenticação de um usuário existente. E também faz a calibragem do reconhecimento biométrico.

\begin{figure}[H]
  \centering
  \caption{Diagrama de Classes.}
  \includegraphics[width=\textwidth]{images/diagrama-de-classes.png}
  \fonte{(AUTORES, 2018)}
  \label{fig:diagrama-de-classes}
\end{figure}




\section{Diagrama ER}
A Figura \ref{fig:diagrama-er} representa a modelagem do banco de dados onde serão persistidos os dados do sistema. O modelo é composto por 5 tabelas principais que são: Usuario, Senha, Biometria, Audio, Imagem. A tabela Usuario mantém o registro dos usuários do sistema, e contém as informações fornecidas no cadastro da conta, ela possui dois relacionamentos com as tabelas Biometria e Senha. A tabela Biometria representa as leituras biometrias coletadas no processo de cadastro, em forma de imagens e gravação de áudio. Estes arquivos são armazenados no dispositivo e seu caminho é salvo no banco de dados referenciando nas tabelas Imagem e Audio de acordo. Na tabela Senha são armazenadas as senhas criptografadas juntamente com informações adicionais do serviço, esta tabela faz parte da principal funcionalidade do sistema que é o armazenamento seguro de senhas. Para cada usuário cadastrado no sistema poderão existir \textit{N} senhas e \textit{N} biometrias. E para cada biometria poderão existir \textit{N} arquivos de áudio e \textit{N} arquivos de imagem.
\begin{figure}[H]
  \centering
  \caption{Diagrama ER.}
  \includegraphics[width=\textwidth]{images/diagrama-er.png}
  \fonte{(AUTORES, 2018)}
  \label{fig:diagrama-er}
\end{figure}



\section{Casos de Uso}
Como pode ser visto na Figura \ref{fig:casos-de-uso}, a modelagem do sistema foi dividida em um conjunto de casos de uso fundamentais para os processos. O único ator envolvido é o próprio usuário, que irá interagir com cada caso de uso, desde operações básicas sobre as senhas, até o cadastro e autenticação biométrica.

\begin{figure}[H]
  \centering
  \caption{Diagrama de Casos de Uso.}
  \includegraphics[width=\textwidth]{casos-de-uso.png}
  \fonte{(AUTORES, 2018)}
  \label{fig:casos-de-uso}
\end{figure}



\subsection{Cadastro Biométrico}
Este caso de uso faz parte do processo principal do sistema, que é a autenticação biométrica. Ele representa o primeiro contado do usuário com a aplicação, onde serão coletadas as leituras biométricas da face do usuário, estas leituras sendo na forma de imagens. As imagens coletadas passam por um processo de normalização, onde são aplicados filtros e redimensionamento, antes de serem armazenadas no dispositivo e referenciadas no banco de dados.
Para se cadastrar no sistema, na primeira etapa o usuário deverá escolher um nome de usuário e uma senha mestre. Esta senha deverá conter no mínimo 10 caracteres, e pelo menos de três diferentes tipos entre: letras minúsculas, letras maiúsculas, dígitos, caracteres especiais. Na etapa seguinte, o usuário deverá informar uma chave que é usada para criptografia das senhas. Esta chave poderá ser gerada automaticamente pelo sistema, e deve ter entre 30 e 60 caracteres. A chave deve ser longa para que seja difícil de ser adivinhada, e possa ser utilizada em diferentes algoritmos de criptografia. E por fim o aplicativo irá solicitar que o usuário aponte a câmera de seu dispositivo para sua face e capture imagens da sua face, enquanto ele demonstra diferentes expressões faciais. Então o sistema de reconhecimento é calibrado automaticamente para reconhecer o usuário baseado em suas características biométricas fornecidas durante o cadastro biométrico. O diagrama de sequência deste caso de uso pode ser visualizado na Figura \ref{fig:cadastro-biometrico}.

\begin{figure}[H]
  \centering
  \caption{Diagrama de Sequência - Cadastro Biométrico.}
  \includegraphics[width=\textwidth]{cadastro-biometrico.png}
  \fonte{(AUTORES, 2018)}
  \label{fig:cadastro-biometrico}
\end{figure}


\subsection{Autenticação biométrica}
A autenticação biométrica é uma das formas de se autenticar no sistema, e estará disponível somente após a conclusão do cadastro biométrico com sucesso. Este processo acontecerá de forma similar ao cadastro biométrico, onde o usuário é solicitado que segure o dispositivo com a câmera apontada para sua face. Este processo deverá acontecer de forma rápida, e com precisão no reconhecimento. O sistema irá realizar o reconhecimento da biometria do usuário, fazer a combinação dos resultados obtidos, e por fim decidir se a autenticação é válida. Em caso positivo, o usuário terá acesso à todas as senhas e informações cadastradas no sistema pelo seu usuário. Caso não seja válido, o usuário poderá tentar o processo novamente.

Durante a autenticação biométrica o reconhecimento facial deve ser possível mesmo em ambientes com pouca luminosidade, e ser independente das expressões faciais. O diagrama de sequência deste caso de uso pode ser visualizado na Figura \ref{fig:autenticacao-biometrica}.

\begin{figure}[H]
  \centering
  \caption{Diagrama de Sequência - Autenticação Biométrica.}
  \includegraphics[width=\textwidth]{autenticacao-biometrica.png}
  \fonte{(AUTORES, 2018)}
  \label{fig:autenticacao-biometrica}
\end{figure}


\subsection{Cadastrar Senha}
O cadastro de senha é uma das operações básicas sobre a principal funcionalidade do sistema de gerenciamento de senha. Este processo faz a persistência de uma nova senha junto com suas informações adicionais no dispositivo. O usuário também pode optar por gerar uma senha automaticamente, onde é informado a quantidade máxima de caracteres que a senha deve conter para que seja gerada uma senha aleatória utilizando caracteres alfanuméricos e especiais. Para cadastrar uma nova senha o usuário deve estar autenticado no sistema. E para que as senhas sejam salvas de forma segura, elas são criptografadas antes de serem salvas, utilizando um algoritmo de criptografia simétrica. A chave utilizada no processo de cifragem é composta de duas partes, uma metade definida pelo usuário na criação de sua conta, e a outra é configurada diretamente no sistema. Esta chave está armazenada na configuração da conta do usuário, e é salva na sessão toda vez que é realizado o login. Então parte desta chave está armazenada no banco de dados, enquanto outra parte está fixa no sistema, a fim de evitar um possível ponto único de falha caso a chave seja obtida através do banco de dados. O diagrama de sequências deste caso de uso pode ser visualizado na Figura \ref{fig:cadastrar-senha}.

\begin{figure}[H]
  \centering
  \caption{Diagrama de Sequência - Cadastrar Senha.}
  \includegraphics[width=\textwidth]{cadastrar-senha.png}
  \fonte{(AUTORES, 2018)}
  \label{fig:cadastrar-senha}
\end{figure}



\subsection{Visualizar Senha}
A visualização das senhas é outra operação básica sobre o gerenciamento de senhas, e o usuário deverá estar autenticado no sistema para poder fazer a recuperação da senha criptografada e visualizá-la em texto claro. Para isto basta o usuário selecionar uma senha que deseja visualizar da lista de senhas cadastradas, esta lista irá mostrar o campo de descrição informado no cadastro da senha. O diagrama de sequência deste caso de uso pode ser visualizado na Figura \ref{fig:visualizar-senha}. A senha que for selecionada para visualização é então encaminhada para ser descriptografada, para isto é utilizado a chave cadastrada na conta do usuário que estará disponível na sessão. Esta chave funciona em conjunto com um algoritmo de criptografia simétrico, onde a mesma é utilizada para criptografia e também para o processo inverso.
\begin{figure}[H]
  \centering
  \caption{Diagrama de Sequência - Visualizar Senha.}
  \includegraphics[width=\textwidth]{visualizar-senha.png}
  \fonte{(AUTORES, 2018)}
  \label{fig:visualizar-senha}
\end{figure}


\subsection{Editar e Excluir Senha}
A edição das senhas é uma combinação dos processos de visualizar e cadastrar senhas, vistos nas Figuras \ref{fig:visualizar-senha} e \ref{fig:cadastrar-senha} respectivamente. O usuário deverá estar autenticado e estar visualizando uma senha par alterá-la. O usuário poderá alterar todas as informações que foram cadastradas junto com a senha, e caso a senha seja alterada ela passará pelo processo de criptografia antes de ser salva novamente.
O processo de excluir uma senha também exige que o usuário esteja autenticado e visualizando uma senha, ele é bastante simples, e somente remove a senha armazenada no dispositivo com a confirmação do usuário.


\section{Diagramas de Atividades}
A Figura \ref{fig:diagrama-de-atividades} representa o diagrama de atividades do sistema. Ele é um gráfico de fluxo que mostra o fluxo de controle de uma atividade para outra, e representa os fluxos conduzidos por processamentos. Nele estão detalhadas as atividades que o usuário pode realizar, e a ordem em que elas ocorrem. O ponto inicial do fluxo é o \textit{login}, onde é realizado o processo de cadastro de um novo usuário e autenticação biométrica para permitir o acesso ou não ao sistema. Em situação de múltiplas tentativas de autenticação sem sucesso, o usuário pode ser bloqueado por um período de tempo. E com o usuário autenticado, é possível utilizar as funcionalidades do sistema de gerenciamento de senhas.

\begin{figure}[H]
  \centering
  \caption{Diagrama de Atividades.}
  \includegraphics[width=\textwidth]{images/diagrama-de-atividades.png}
  \fonte{(AUTORES, 2018)}
  \label{fig:diagrama-de-atividades}
\end{figure}


\chapter{Sistema desenvolvido}
\label{sistema}

Este capítulo contempla o desenvolvimento do sistema de armazenamento de senhas, ele foi desenvolvido seguindo a modelagem especificada no capítulo 5, e cada uma das etapas estão descritas nas seções seguintes.


\section{Tecnologias}
O desenvolvimento do sistema de gerenciamento de senhas foi na linguagem Python, pois ela possui bibliotecas de código livre que implementam diversos algoritmos para processamento de imagens, reconhecimento de características biométricas e criptografia, incluindo os que foram escolhidos para o desenvolvimento deste trabalho. O sistema foi desenvolvido para a plataforma web, pois ela permite que a aplicação seja disponibilizada e utilizada em diversos tipos de dispositivos, como \textit{smartphones}, \textit{notebooks}, e computadores \textit{desktop}. Para utilizar o sistema é preciso ter instalado um navegador \textit{web} que possa acessar as \textit{API's} da câmera e/ou áudio do dispositivo, que é possível na grande maioria dos navegadores atuais. O sistema desenvolvido segue a modelagem apresentada na seção 5.3 e foi dividido em partes duas partes.
Todas as informações cadastradas no sistema são armazenadas em um banco de dados relacional. Especificamente neste trabalho foram utilizados dois bancos distintos, um para ambientes de teste que é o \textit{SQLite}, e um para ambientes de produção que é o \textit{MySQL}. E todas as informações sensíveis, como senhas e usuários são armazenadas de forma criptografada, onde somente usuários autenticados podem acessá-las em forma de texto.

O \textit{hardware} utilizado para o desenvolvimento foi um computador pessoal com a IDE (Ambiente de desenvolvimento integrado) PyCharm e Python instalados. Também foi utilizado o navegador web \textit{Chrome} para realizar testes e simulações da aplicação durante o desenvolvimento. O sistema foi construído visando a modularidade, para que seja facilmente integrado novos algoritmos de criptografia e novas biometrias na autenticação, além de outras funcionalidades adicionais que podem ser propostas em trabalhos futuros.


\section{Autenticação biométrica}
O sistema de gerenciamento de senhas utiliza técnicas de reconhecimento biométrico para autenticar os usuários, sendo está o reconhecimento de faces. O processo exige uma calibragem em seu primeiro uso, e funcionam comparando leituras biométricas atuais com leituras utilizadas no treinamento do algoritmo. Para o reconhecimento facial foi escolhida a técnica de \textit{FisherFaces}, pois analisando os trabalhos relacionados ela demonstrou as melhores taxas de reconhecimento, que chegam a aproximadamente 95\%, como é possível ver na Figura \ref{fig:reconhecimento-facial} citada no Capítulo 2. E esta técnica é implementada pelo \textit{OpenCV} \cite{opencv}, uma das tecnologias que foi estudada e utilizada para o desenvolvimento do sistema.

A autenticação através do reconhecimento facial é feita utilizando o algoritmo \textit{Fisherfaces} da biblioteca \textit{OpenCV} disponível em Python e C++. A implementação do reconhecimento facial é dividida em duas partes, a coleta de imagens e treinamento do algoritmo, e a autenticação através do reconhecimento da face do usuário com o algoritmo treinado. Para cada etapa são utilizadas técnicas de pré-processamento de imagens para prepara-las para o algoritmo.
O algoritmo \textit{Fisherfaces} é um algoritmo de aprendizado de máquina bastante complexo, e para que ele seja treinado corretamente é necessário que as imagens passem por várias etapas de pré-processamento. Primeiramente o algoritmo espera que as imagens estejam em escala de cinza, com as mesmas dimensões, e que sejam da face enquadrada da pessoa. 

Para obter estes requisitos, foram utilizados classificadores \textit{Haar feature-based cascade} proposto por Paul Viola and Michael Jones \cite{violajones:2001}. Eles são implementados pelo \textit{openCV} e foram utilizados dois classificadores pré treinados no processo, um para faces e outro para olhos. É possível ver a aplicação destes classificadores na detecção de faces e olhos na Figura \ref{fig:haarcascade-opencv}.

\begin{figure}[H]
  \centering
  \caption{Haar feature-based cascade.}
  \includegraphics[width=100mm]{images/haarcascade-opencv.png}
  \fonte{\cite{opencv}}
  \label{fig:haarcascade-opencv}
\end{figure}

O sistema conta com uma seção de configuração onde o usuário pode capturar entre 20 e 50 fotos contendo sua face, que são enviadas para o servidor da aplicação web. O primeiro processo é converter as imagens para escala de cinza, e então aplicar o detector de face na imagem, quanto a face é encontrada então é aplicado ainda o detector de olhos para garantir que a face detectada realmente é uma face humana e não alguma inconsistência do algoritmo. Após a confirmação da face detectada, ela é recortada da imagem inicial e então redimensionada em um padrão de 200x200 \textit{pixels}, pois o algoritmo exige que todas imagens possuam as mesmas dimensões. Este procedimento é feito para todas as imagens que são armazenadas em uma lista. Esta lista é passada para o algoritmo de treinamento juntamente com uma segunda lista com a mesma quantidade de elementos contendo a classe de cada imagem respectivamente. No caso desta implementação contém só uma classe que é a classe 1 do usuário.

O resultado obtido pelo treinamento do algoritmo \textit{Fisherfaces} é um arquivo contendo as características extraídas das imagens e que representa o seu treinamento. Este arquivo é utilizado para realizar o reconhecimento das faces na autenticação biométrica. O procedimento de reconhecimento da face é semelhante ao do treinamento, onde a imagem a ser reconhecida passa pelas mesmas técnicas de pré-processamento e utilizada pelo algoritmo juntamente com o arquivo gerado pelo treinamento para obter a classificação. O retorno do algoritmo são dois parâmetros, a classe da imagem reconhecida e a confiança do reconhecimento. A classe é referente a classe informada no processo de treinamento, e caso a face não seja reconhecida é retornado uma constante -1, e a confiança é um número que representa a distância entre a imagem a ser reconhecida e as imagens do treinamento. Quanto mais próximo a zero, mais a imagem se assemelha as utilizadas no treinamento.

\section{Cadastro de senhas}
Quando o usuário é autenticado com sucesso, através do usuário e biometria ou senha, ele é redirecionado para a tela principal, onde está a principal funcionalidade do sistema, o armazenamento seguro das senhas. Após a autenticação do usuário é possível acessar a tela principal, onde são estão cadastradas as senhas e informações adicionais do usuário. É possível editar excluir e adicionar novas senhas. Para cada senha pode ser informado um usuário, a senha em si, a \textit{URL} do serviço ou \textit{website} referente e uma descrição. Estas informações só estão visíveis em forma de texto claro, pois o usuário está autenticado no sistema. 

Caso contrário elas não podem ser acessadas, e são armazenadas em um banco de dados em forma de texto cifrado. O processo de criptografia é feito utilizando uma chave composta, onde uma parte da chave é informada pelo usuário ao criar sua conta, e a outra metade é definida internamente pelo sistema. Cada vez que o usuário é autenticado com sucesso, as senhas são então decifradas utilizando esta chave composta. A cifragem das senhas é a principal forma de segurança do sistema e garantem que mesmo que o banco de dados seja obtido por pessoas mal-intencionadas, elas sejam extremamente difíceis de serem decifradas.

O sistema possui uma funcionalidade importante que serve para encaminhar o seu uso da forma correta, esta funcionalidade é a geração de senhas automaticamente. Ao adicionar uma nova senha, o usuário tem a possibilidade informar a senha, ou gerar uma aleatoriamente informando o número máximo de caracteres que a esta poderá conter, por padrão este número está definido em 15 caracteres. Como dito, as senhas geradas são complexas e não consideram nenhuma informação ou característica do usuário. Estas senhas geradas são consideradas complexas pois são compostas por diferentes tipos de caracteres como, letras maiúsculas e minúsculas, dígitos e caracteres especiais.



O uso desta funcionalidade é extremamente recomendado aos usuários, pois ele é uma da justificativa a uma das principais razões para se utilizar um sistema de armazenamento de senhas, que é a utilização de senhas que sejam complexas, longas e difíceis de se memorizar. Então, é interessante que ao ir se registrar em um novo serviço ou \textit{website} na internet, o usuário utilize o sistema para criar uma nova senha, e ele não precisa utilizar uma senha repetida ou ter de memorizar a mesma.



\section{Armazenamento seguro de senhas}
Para aumentar a segurança da aplicação e proteger os dados dos usuários, foram aplicados as técnicas e algoritmos de criptografia estudados. Basicamente foram utilizadas duas diferentes formas de criptografia, a simétrica para as informações sobre as senhas cadastradas no sistema, e assimétrica para cifrar a senha do usuário do sistema efetivamente. 

O sistema desenvolvido utiliza biometria para autenticação, porém ainda é necessário que o usuário possua uma senha mestre para casos como, o primeiro uso do sistema antes de realizar a calibragem das biometrias, ou caso o seu dispositivo não possua meios de capturar suas biometrias e o usuário precisa de suas informações. Para criptografar esta senha foi utilizado o algoritmo SHA256, que é um algoritmo assimétrico que realiza a criptografia da senha, e não é capaz de realizar o processo inverso. O algoritmo utilizado é implementado pela biblioteca \textit{passlib} em Python, e possui algumas características que aumentam sua segurança. Como pode ser visto na figura \ref{fig:sha256} o algoritmo é utilizado para criptografar a mesma palavra (\textit{password}), porém ele não resulta na mesma palavra criptografada. Isto ocorre pois ele utiliza uma técnica de \textit{salt}, onde é utilizado informações adicionais do sistema para acrescentar uma sequência extra de caracteres no resultado criptografia. Isto o torna mais robusto contra-ataques do tipo \textit{hashtables} \cite{schneier:2007}, que são tabelas que contém palavras mapeadas a sua versão criptografada. Então a senha mestre do usuário é criptografada utilizando o algoritmo SHA256 e a cada tentativa de autenticação sua senha informada é criptografada e então comparada com a armazenada no banco. O algoritmo consegue verificar se elas vêm da mesma palavra mesmo gerando 
\textit{hashs} diferentes.

\begin{figure}[H]
  \centering
  \caption{Algoritmo SHA256.}
  \includegraphics[width=\textwidth]{images/sha256.png}
  \fonte{(AUTORES, 2018)}
  \label{fig:sha256}
\end{figure}

Já para a criptografia das informações de senhas cadastradas pelo usuário, é utilizado uma técnica de criptografia simétrica, pois estas informações precisam ser decifradas quando o usuário deseja visualizá-las. O algoritmo utilizado foi o AES, implementado pela biblioteca \textit{PyCrypto}. O algoritmo utiliza uma chave única, que é utilizada na cifragem e decifragem de textos. Para aumentar a segurança do sistema a chave utilizada no algoritmo é composta de duas partes, uma que é informada pelo usuário durante o seu cadastro, e outra que é mantida pelo sistema. O conjunto das duas gera uma chave única que então é utilizada com o AES. A cada autenticação com sucesso as senhas e informações adicionais do usuário são decifradas e podem ser visualizadas em forma de texto.





\section{Acesso as senhas}
Antes de acessar o sistema é preciso criar um novo usuário, para isto existe uma tela onde são preenchidas suas informações e então criado o usuário. A criação de um usuário não envolve o cadastro e calibragem do reconhecimento facial, pois isto é feito após o primeiro acesso ao sistema. Como pode ser visto na Figura \ref{fig:cadastro-usuario}, no cadastro de um novo usuário é preciso informar um nome de usuário, e-mail, a senha e sua confirmação, assim como uma chave que é utilizada parcialmente para a criptografia das senhas cadastradas futuramente. Também é validado se a senha informada é considerada uma senha forte, contendo caracteres variados como letras maiúsculas e minúsculas, dígitos e caracteres especiais, além de não ser muito curta, pois esta senha é importante para o usuário e não deve ser uma senha simples que pode ser adivinhada através de tentativa e erro.

\begin{figure}[H]
  \centering
  \caption{Cadastro de novo usuário.}
  \includegraphics[width=\textwidth]{images/cadastro-usuario.png}
  \fonte{(AUTORES, 2018)}
  \label{fig:cadastro-usuario}
\end{figure}


Para acessar o sistema é preciso primeiramente ter criado um usuário, então basta utilizar as informações de usuário e senha para acessar o sistema. Na Figura \ref{fig:login} é possível visualizar a tela de login, onde existem dois campos, um para o usuário e outro para a senha, e uma área de visualização previa da câmera do dispositivo. A captura da imagem da câmera serve para o processo de reconhecimento biométrico e para possibilitar o seu uso, é preciso conceder as permissões necessárias para que o usuário possa se autenticar utilizando o reconhecimento facial. Então a autenticação pode ser feita utilizando o usuário e senha cadastrados, ou o usuário e a sua face. Após pressionar o botão entrar, o sistema busca pelo usuário informado, se foi informado uma senha, é feito o login utilizando a mesma, caso contrário é utilizado o reconhecimento biométrico da face capturada no momento em que foi pressionado o botão. Se o usuário e/ou senha estiverem incorretos, ou a face não for reconhecida, o usuário é redirecionado novamente para a tela de login e é apresentado uma mensagem de erro não específica sobre a falha na autenticação.

\begin{figure}[H]
  \centering
  \caption{Tela de login.}
  \includegraphics[width=\textwidth]{images/nao-reconhecida_censored.jpg}
  \fonte{(AUTORES, 2018)}
  \label{fig:login}
\end{figure}







\chapter{Resultados}
\label{resultados}

Este Capítulo contempla os testes que foram aplicados durante o desenvolvimento do sistema de armazenamento de senhas. Foi feito um teste completo das funcionalidades do sistema e dos algoritmos de reconhecimento facial aplicados na biometria, visando os requisitos definidos na modelagem.


\section{Procedimentos metodológicos}
O procedimento de validação do sistema é qualitativo, onde foi feita uma síntese dos resultados obtidos, evidenciando as contribuições, relatando as limitações do estudo, relacionando os fatos verificados com a teoria. Foram feitos testes sobre o processo de autenticação através do uso de biometria e algoritmos de aprendizado de máquina, além de testes com usuários legítimos procurando falhas no processo de autenticação, e também usuários não legítimos tentando burlar a segurança no login, utilizando características biométricas forjadas com imagens impressas ou digitalizadas.


\section{Cadastro de usuário e login}
Primeiramente foi feito o cadastro de um usuário preenchendo os campos no cadastro, observou-se que a senha do usuário deve ser uma senha considerada forte. Esta força aumenta de acordo com o seu tamanho a variedade de tipos de caracteres utilizados. O campo de chave exige que sejam informados 16 caracteres, porém não é considerado a complexidade desta chave. Após criado o novo usuário, foi feito a tentativa de login com o mesmo, que ocorreu normalmente, quando informados o usuário e senha corretos, e o sistema foi redirecionado para a tela principal onde está disponível o cadastro das senhas, como pode ser observado na Figura \ref{fig:senhas}. 

\begin{figure}[H]
  \centering
  \caption{Tela principal do sistema de armazenamento de senhas.}
  \includegraphics[width=\textwidth]{images/login-sucesso.png}
  \fonte{(AUTORES, 2018)}
  \label{fig:senhas}
\end{figure}

Para validar o processo de autenticação também foi feito a tentativa de login utilizando usuário ou senha incorretos, neste caso o sistema não fez a autenticação e retornou uma mensagem informando o erro. Após múltiplas tentativas de login com erro, o sistema retorna uma mensagem informando que o usuário foi bloqueado temporariamente por 5 minutos, como pode ser observado na Figura \ref{fig:usuario-bloqueado}. 

\begin{figure}[H]
  \centering
  \caption{Usuário bloqueado temporariamente após tentativas de login sem sucesso.}
  \includegraphics[width=\textwidth]{images/usuario-bloqueado_censored.png}
  \fonte{(AUTORES, 2018)}
  \label{fig:usuario-bloqueado}
\end{figure}



\section{Cadastro e armazenamento de senhas}
Com o usuário autenticado foram feitos inserções de senhas e suas informações adicionais, observou-se que ao criar uma nova senha o sistema oferece a possibilidade da geração automática de uma senha informando o número de caracteres que ela deve conter. As senhas geradas possuem diferentes tipos de caracteres, são complexas e aleatórias, como pode ser observado na Figura \ref{fig:editar-senhas}.

\begin{figure}[H]
  \centering
  \caption{Cadastro e geração automática de senhas.}
  \includegraphics[width=\textwidth]{images/editar-senha.png}
  \fonte{(AUTORES, 2018)}
  \label{fig:editar-senhas}
\end{figure}

Após salvar a senha, as informações são inseridas no banco de dados, podendo ser editadas e excluídas. Para verificar se as senhas foram armazenadas de forma segura foram acessados o banco de dados da aplicação simulando uma possível falha do sistema onde este banco pudesse ser acessado. Como pode ser observado na Figura \ref{fig:senhas-cifradas}, as informações de usuário, senha, URL e descrição estão salvas no banco em forma de texto cifrado, e não podem ser descobertas facilmente sem a utilização do algoritmo para realizar a decifragem juntamente com a chave utilizada para o processo.


\begin{figure}[H]
  \centering
  \caption{Senhas criptografadas armazenadas no banco de dados.}
  \includegraphics[width=\textwidth]{images/senha-cifrada.png}
  \fonte{(AUTORES, 2018)}
  \label{fig:senhas-cifradas}
\end{figure}


\section{Cadastro e calibragem de biometria}
Como o sistema dá a possibilidade ao usuário de utilizar a autenticação biométrica por reconhecimento facial, foram feitos testes também desta funcionalidade além dos testes dos algoritmos utilizados já apresentados. Na tela de configuração, foram capturadas imagens da face do usuário autenticado, o sistema exigiu pelo menos 20 imagens para possibilitar o treinamento. Pode perceber-se que o sistema não valida se as imagens realmente são da mesma pessoa, ou contém a face do usuário. Porém para simplicidade dos testes foram utilizadas imagens integras todas contendo a face do usuário, e variações leves de expressões faciais e angulo. A tela de cadastro da biometria que pode ser observada na Figura \ref{fig:cadastro-biometrico} conseguiu capturar até 50 imagens, e as imagens após esse número começam a substituir as primeiras formando um ciclo, e o botão limpar apaga todas as imagens já capturadas. Após a captura das imagens, elas foram submetidas para o treinamento do reconhecimento. Com o treinamento feito o sistema retornou uma mensagem dizendo que a biometria já está cadastrada e o usuário já pode utilizar o reconhecimento facial para login.

\begin{figure}[H]
  \centering
  \caption{Tela de cadastro e calibragem de biometria.}
  \includegraphics[width=\textwidth]{images/configuracao_censored.jpg}
  \fonte{(AUTORES, 2018)}
  \label{fig:cadastro-biometrico}
\end{figure}


\section{Autenticação biométrica}
Para validar a autenticação biométrica foi feito a tentativa de login utilizando as leituras da biometria da face do usuário através de uma imagem. Na tela de login o sistema pede a permissão para utilizar a câmera do dispositivo, e apresenta a visualização prévia da câmera junto com os campos de usuário e senha. Ao tentar logar diretamente com a câmera apontando para a face, o sistema exige que seja informado pelo menos o nome de usuário. Com o usuário preenchido o sistema foi capaz de reconhecer rapidamente e realizar a autenticação no sistema. Então foram feitos testes utilizando um nome de usuário correto, porém com outra pessoa aparecendo na câmera. O sistema não foi capaz de fazer a autenticação e retornou a mensagem de erro respectiva. O login através do reconhecimento facial também apresenta erro quando a face não está claramente aparecendo na imagem, foram testados para exemplo as seguintes formas: face não aparece completamente, ambiente muito escuro, óculos ou objetos sobre a face.

Também foram aplicados os testes de login com usuário não legítimo através de biometrias forjadas, e o sistema demonstrou uma fraqueza, sendo capaz de reconhecer em algumas tentativas o usuário através de imagens impressas, ou digitais em uma tela de \textit{smartphone} ou tablet, como pode ser observado na Figura \ref{fig:biometria-forjada}.

\begin{figure}[H]
  \centering
  \caption{Login utilizando biometria na tela de um \textit{smartphone}.}
  \includegraphics[width=\textwidth]{images/biometria-forjada_censored.png}
  \fonte{(AUTORES, 2018)}
  \label{fig:biometria-forjada}
\end{figure}


\section{Reconhecimento facial}
O algoritmo para reconhecimento facial escolhido para ser utilizado no sistema foi o \textit{Fisherfaces}, ele está disponível na biblioteca \textit{OpenCV} para Python e C++, juntamente com outros algoritmos de reconhecimento facial como \textit{Eigenfaces} e LBPH(\textit{Local Binary Patterns Histograms}). Para medir a capacidade de reconhecimento do algoritmo utilizado no reconhecimento de faces em imagens, foi aplicado a técnica de validação de resultados \textit{K-Fold Cross Validation (KFCV)}.
Esta técnica consiste em dividir a base em K conjuntos, onde um destes conjuntos é utilizado para testes enquanto os restantes são utilizados para o treinamento do algoritmo. Segundo \cite{kaufmann:2006}, a técnica de validação cruzada utilizando 10 \textit{folds} é recomendado para estimar a acurácia de um classificador, mesmo que os computadores atuais sejam capazes de utilizar mais \textit{folds}.

Os testes do algoritmo foram feitos em Python, onde foram utilizadas as técnicas estudadas junto com o método de validação cruzada. As bases de testes utilizadas foram a \textit{Yale Faces Database} \cite{yale:2001} que é uma base de faces que contém no total 165 imagens em escala de cinza estas imagens são de 15 indivíduos distintos, onde para cada um existem 11 imagens com diferentes expressões faciais, e expostas a diferentes condições de luminosidade, como pode ser observado na figura \ref{fig:yalefaces}.

\begin{figure}[H]
  \centering
  \caption{\textit{Yale Faces Database}.}
  \includegraphics[width=\textwidth]{images/yalefaces.png}
  \fonte{\cite{yale:2001}}
  \label{fig:yalefaces}
\end{figure}

Enquanto a segunda base utilizada foi \textit{ORL Database of Faces} que contém imagens um total de  de 400 imagens em escala de cinza de 40 indivíduos distintos, e para cada um destes existem 10 imagens com diferentes expressões faciais, e expostas a diferentes condições de luminosidade, similares a base de Yale.

O procedimento do teste aplicado foi semelhante para ambas as bases, e ocorreu da seguinte maneira. Foi criado uma pasta contendo todas as imagens de faces, e para cada individuo suas imagens foram agrupadas dentro de outra pasta com seu respectivo código. Então foi aplicado a sequência de testes, que são divididas em 10 etapas. Inicialmente a base é dividida em 10 conjuntos iguais de imagens aleatoriamente, então para cada etapa, 9 dos 10 conjuntos são utilizados para treinamento e o restante para testes, assim até que todos os conjuntos tenham sido utilizados para teste e para treinamento. Os resultados podem ser visualizados na \ref{tab:fisherface}.

\begin{table}[H]
\centering
\caption{Resultados testes Fisherfaces.}
\label{tab:fisherface}
\begin{tabular}{|l|r|l|}
\hline
\textbf{Base}                  & \textbf{Taxa de acertos} & \textbf{Parâmetros} \\ \hline
\textit{Yale Face Database}    & 96,67\% & \texttt{num\_components=14}  \\ \hline
\textit{ORL Database of Faces} & 95,75\% & \texttt{num\_components=39}  \\ \hline
\end{tabular}
\fonte{(AUTORES, 2018)}
\end{table}

O algoritmo também pode receber 2 parâmetros opcionais, que são: \textit{num\_components} e \textit{threshold}. Caso estes parâmetros não sejam fornecidos são utilizados os valores padrão predefinidos pela biblioteca. O número de componentes é uma forma de aumentar a quantidade de características que serão extraídas da face, enquanto o \textit{threshhold} é o limite mínimo de confiança que deve ser considerado para uma face ser reconhecida. Estes parâmetros não tiveram muitos efeitos sobre os testes, mas foram utilizados na implementação do sistema de gerenciamento de senhas para melhorar a calibragem do treinamento do algoritmo \textit{Fisherfaces}. O algoritmo \textit{Fisherfaces} demonstra altos índices de reconhecimento até mesmo em imagens com pouca luminosidade o que é uma de suas principais características. E sua performance é constante nas duas bases, o que significa que ele é robusto e capaz de reconhecer faces de forma eficiente.


Também foram feitos testes com o algoritmo de reconhecimento facial \textit{Eigenfaces}, que também é implementado pelo \textit{OpenCV}, e ele funciona de maneira muito parecida com o \textit{Fisherfaces}. Para procedimento de testes utilizou-se as mesmas bases, as mesmas etapas de pre-processamento, e a mesma técnica de validação anteriores. Os resultados obtidos podem ser visualizados na tabela \ref{tab:eigenfaces}. Pode perceber que o algoritmo também possui um desempenho excelente, e inclusive é capaz de superar o seu concorrente em uma das bases de teste. Isto ocorre pois a base \textit{ORL Database of Faces}, contém imagens com boas condições de luminosidade, enquanto na \textit{Yale Face Database}, existem variações, e é uma característica do algoritmo possuir um baixo desempenho em situações de pouca luminosidade.

\begin{table}[H]
\centering
\caption{Resultados testes Eigenfaces.}
\label{tab:eigenfaces}
\begin{tabular}{|l|r|l|}
\hline
\textbf{Base}                  & \textbf{Taxa de acertos} & \textbf{Parâmetros} \\ \hline
\textit{Yale Face Database}    & 83,33\% & \texttt{num\_components=149}  \\ \hline
\textit{ORL Database of Faces} & 97,50\% & \texttt{num\_components=359}  \\ \hline
\end{tabular}
\fonte{(AUTORES, 2018)}
\end{table}



Porém para autenticar um usuário no sistema é importante definir um \textit{threshold} mínimo, para evitar que ocorra reconhecimento falsos positivos, onde o sistema considera uma face desconhecida como sendo um usuário cadastrado. Para isto foram feitos testes de reconhecimento para verificar aproximadamente até quanto esse limiar pode ser considerado válido. E através destes testes pode-se perceber que quando a confiança do reconhecimento está muito alta, significa que o algoritmo não foi capaz de identificar com tanta precisão a face e pode ter generalizado uma face desconhecida como sendo a de um usuário do sistema. Para isto foi definido o limite de confiança em aproximadamente 2000, que é o limite mínimo considerado para efetivar a autenticação.
\chapter{Conclusão e Trabalhos Futuros}
\label{conclusao}

O presente trabalho buscou projetar e desenvolver um sistema de gerenciamento de senhas seguro utilizando autenticação biométrica e algoritmos de criptografia. A partir da revisão bibliográfica foi estudado o estado da arte da autenticação biométrica, e identificar quais são suas vantagens e também suas limitações. Também foram pesquisados trabalhos relacionados ao reconhecimento biométrico aplicado em sistemas de gerenciamento de senhas.

Para atingir este objetivo foram estudados trabalhos relacionados onde foram aplicados técnicas de reconhecimento biométrico sobre diversos tipos de biometria. Foram levantados as biometrias e técnicas de reconhecimento utilizadas em cada trabalho e foram comparados seus resultados a fim de definir as biometrias e técnicas a serem aplicadas neste trabalho. Então para o desenvolvimento do sistema de gerenciamento de senhas, foi explorado o uso de características biométricas que podem ser obtidas através de computadores e dispositivos móveis e que possuem bons resultados já comprovados, no caso deste trabalho: faces. 

Foi desenvolvido um sistema web com as funcionalidades de inserção, geração e manipulação de senhas e informações adicionais, que são armazenadas de forma segura utilizando algoritmos de criptografia. Para acessar estas funcionalidades e utilizar o sistema é preciso fazer o cadastro biométrico, e ser autenticado utilizando a face ou a senha mestre. As técnicas de reconhecimento facial utilizadas no sistema demonstram boas taxas de reconhecimento em testes realizados em bases de faces, obtendo aproximadamente 96\% de precisão no reconhecimento.

O sistema desenvolvido apresentou diferentes características presentes nos trabalhos relacionados estudados, e também a soma destas características com novas funcionalidades, que no caso deste trabalho foi o geração e armazenamento seguro de senhas, e a autenticação biométrica por reconhecimento facial.


Para trabalhos futuros, sugere-se explorar outras características biométricas que foram estudadas neste trabalho, porém não foram aplicadas ao sistema, como voz e íris, que possa deixar o sistema ainda mais seguro. Outro ponto a ser estudado é a avaliação da qualidade das leituras biométricas, e atribuição de pesos para cada biometria em um sistema de reconhecimento biométrico multimodal.

Existem diversas técnicas para reconhecimento de características biométricas, como faces e voz. Este trabalho abordou duas destas, \textit{Fisherfaces} e \textit{Eigenfaces}. Trabalhos futuros podem possibilitar de configurar a escolha da técnica a ser utilizada para o reconhecimento de cada biometria.


\bibliography{referencias}

% Anexos

\end{document}
