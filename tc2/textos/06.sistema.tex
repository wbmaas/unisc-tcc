\chapter{Sistema desenvolvido}
\label{sistema}

Este capítulo contempla o desenvolvimento do sistema de armazenamento de senhas, ele foi desenvolvido seguindo a modelagem especificada no capítulo 5, e cada uma das etapas estão descritas nas seções seguintes.


\section{Tecnologias}
O desenvolvimento do sistema de gerenciamento de senhas foi na linguagem Python, pois ela possui bibliotecas de código livre que implementam diversos algoritmos para processamento de imagens, reconhecimento de características biométricas e criptografia, incluindo os que foram escolhidos para o desenvolvimento deste trabalho. O sistema foi desenvolvido para a plataforma web, pois ela permite que a aplicação seja disponibilizada e utilizada em diversos tipos de dispositivos, como \textit{smartphones}, \textit{notebooks}, e computadores \textit{desktop}. Para utilizar o sistema é preciso ter instalado um navegador \textit{web} que possa acessar as \textit{API's} da câmera e/ou áudio do dispositivo, que é possível na grande maioria dos navegadores atuais. O sistema desenvolvido segue a modelagem apresentada na seção 5.3 e foi dividido em partes duas partes.
Todas as informações cadastradas no sistema são armazenadas em um banco de dados relacional. Especificamente neste trabalho foram utilizados dois bancos distintos, um para ambientes de teste que é o \textit{SQLite}, e um para ambientes de produção que é o \textit{MySQL}. E todas as informações sensíveis, como senhas e usuários são armazenadas de forma criptografada, onde somente usuários autenticados podem acessá-las em forma de texto.

O \textit{hardware} utilizado para o desenvolvimento foi um computador pessoal com a IDE (Ambiente de desenvolvimento integrado) PyCharm e Python instalados. Também foi utilizado o navegador web \textit{Chrome} para realizar testes e simulações da aplicação durante o desenvolvimento. O sistema foi construído visando a modularidade, para que seja facilmente integrado novos algoritmos de criptografia e novas biometrias na autenticação, além de outras funcionalidades adicionais que podem ser propostas em trabalhos futuros.


\section{Autenticação biométrica}
O sistema de gerenciamento de senhas utiliza técnicas de reconhecimento biométrico para autenticar os usuários, sendo está o reconhecimento de faces. O processo exige uma calibragem em seu primeiro uso, e funcionam comparando leituras biométricas atuais com leituras utilizadas no treinamento do algoritmo. Para o reconhecimento facial foi escolhida a técnica de \textit{FisherFaces}, pois analisando os trabalhos relacionados ela demonstrou as melhores taxas de reconhecimento, que chegam a aproximadamente 95\%, como é possível ver na Figura \ref{fig:reconhecimento-facial} citada no Capítulo 2. E esta técnica é implementada pelo \textit{OpenCV} \cite{opencv}, uma das tecnologias que foi estudada e utilizada para o desenvolvimento do sistema.

A autenticação através do reconhecimento facial é feita utilizando o algoritmo \textit{Fisherfaces} da biblioteca \textit{OpenCV} disponível em Python e C++. A implementação do reconhecimento facial é dividida em duas partes, a coleta de imagens e treinamento do algoritmo, e a autenticação através do reconhecimento da face do usuário com o algoritmo treinado. Para cada etapa são utilizadas técnicas de pré-processamento de imagens para prepara-las para o algoritmo.
O algoritmo \textit{Fisherfaces} é um algoritmo de aprendizado de máquina bastante complexo, e para que ele seja treinado corretamente é necessário que as imagens passem por várias etapas de pré-processamento. Primeiramente o algoritmo espera que as imagens estejam em escala de cinza, com as mesmas dimensões, e que sejam da face enquadrada da pessoa. 

Para obter estes requisitos, foram utilizados classificadores \textit{Haar feature-based cascade} proposto por Paul Viola and Michael Jones \cite{violajones:2001}. Eles são implementados pelo \textit{openCV} e foram utilizados dois classificadores pré treinados no processo, um para faces e outro para olhos. É possível ver a aplicação destes classificadores na detecção de faces e olhos na Figura \ref{fig:haarcascade-opencv}.

\begin{figure}[H]
  \centering
  \caption{Haar feature-based cascade.}
  \includegraphics[width=100mm]{images/haarcascade-opencv.png}
  \fonte{\cite{opencv}}
  \label{fig:haarcascade-opencv}
\end{figure}

O sistema conta com uma seção de configuração onde o usuário pode capturar entre 20 e 50 fotos contendo sua face, que são enviadas para o servidor da aplicação web. O primeiro processo é converter as imagens para escala de cinza, e então aplicar o detector de face na imagem, quanto a face é encontrada então é aplicado ainda o detector de olhos para garantir que a face detectada realmente é uma face humana e não alguma inconsistência do algoritmo. Após a confirmação da face detectada, ela é recortada da imagem inicial e então redimensionada em um padrão de 200x200 \textit{pixels}, pois o algoritmo exige que todas imagens possuam as mesmas dimensões. Este procedimento é feito para todas as imagens que são armazenadas em uma lista. Esta lista é passada para o algoritmo de treinamento juntamente com uma segunda lista com a mesma quantidade de elementos contendo a classe de cada imagem respectivamente. No caso desta implementação contém só uma classe que é a classe 1 do usuário.

O resultado obtido pelo treinamento do algoritmo \textit{Fisherfaces} é um arquivo contendo as características extraídas das imagens e que representa o seu treinamento. Este arquivo é utilizado para realizar o reconhecimento das faces na autenticação biométrica. O procedimento de reconhecimento da face é semelhante ao do treinamento, onde a imagem a ser reconhecida passa pelas mesmas técnicas de pré-processamento e utilizada pelo algoritmo juntamente com o arquivo gerado pelo treinamento para obter a classificação. O retorno do algoritmo são dois parâmetros, a classe da imagem reconhecida e a confiança do reconhecimento. A classe é referente a classe informada no processo de treinamento, e caso a face não seja reconhecida é retornado uma constante -1, e a confiança é um número que representa a distância entre a imagem a ser reconhecida e as imagens do treinamento. Quanto mais próximo a zero, mais a imagem se assemelha as utilizadas no treinamento.

\section{Cadastro de senhas}
Quando o usuário é autenticado com sucesso, através do usuário e biometria ou senha, ele é redirecionado para a tela principal, onde está a principal funcionalidade do sistema, o armazenamento seguro das senhas. Após a autenticação do usuário é possível acessar a tela principal, onde são estão cadastradas as senhas e informações adicionais do usuário. É possível editar excluir e adicionar novas senhas. Para cada senha pode ser informado um usuário, a senha em si, a \textit{URL} do serviço ou \textit{website} referente e uma descrição. Estas informações só estão visíveis em forma de texto claro, pois o usuário está autenticado no sistema. 

Caso contrário elas não podem ser acessadas, e são armazenadas em um banco de dados em forma de texto cifrado. O processo de criptografia é feito utilizando uma chave composta, onde uma parte da chave é informada pelo usuário ao criar sua conta, e a outra metade é definida internamente pelo sistema. Cada vez que o usuário é autenticado com sucesso, as senhas são então decifradas utilizando esta chave composta. A cifragem das senhas é a principal forma de segurança do sistema e garantem que mesmo que o banco de dados seja obtido por pessoas mal-intencionadas, elas sejam extremamente difíceis de serem decifradas.

O sistema possui uma funcionalidade importante que serve para encaminhar o seu uso da forma correta, esta funcionalidade é a geração de senhas automaticamente. Ao adicionar uma nova senha, o usuário tem a possibilidade informar a senha, ou gerar uma aleatoriamente informando o número máximo de caracteres que a esta poderá conter, por padrão este número está definido em 15 caracteres. Como dito, as senhas geradas são complexas e não consideram nenhuma informação ou característica do usuário. Estas senhas geradas são consideradas complexas pois são compostas por diferentes tipos de caracteres como, letras maiúsculas e minúsculas, dígitos e caracteres especiais.



O uso desta funcionalidade é extremamente recomendado aos usuários, pois ele é uma da justificativa a uma das principais razões para se utilizar um sistema de armazenamento de senhas, que é a utilização de senhas que sejam complexas, longas e difíceis de se memorizar. Então, é interessante que ao ir se registrar em um novo serviço ou \textit{website} na internet, o usuário utilize o sistema para criar uma nova senha, e ele não precisa utilizar uma senha repetida ou ter de memorizar a mesma.



\section{Armazenamento seguro de senhas}
Para aumentar a segurança da aplicação e proteger os dados dos usuários, foram aplicados as técnicas e algoritmos de criptografia estudados. Basicamente foram utilizadas duas diferentes formas de criptografia, a simétrica para as informações sobre as senhas cadastradas no sistema, e assimétrica para cifrar a senha do usuário do sistema efetivamente. 

O sistema desenvolvido utiliza biometria para autenticação, porém ainda é necessário que o usuário possua uma senha mestre para casos como, o primeiro uso do sistema antes de realizar a calibragem das biometrias, ou caso o seu dispositivo não possua meios de capturar suas biometrias e o usuário precisa de suas informações. Para criptografar esta senha foi utilizado o algoritmo SHA256, que é um algoritmo assimétrico que realiza a criptografia da senha, e não é capaz de realizar o processo inverso. O algoritmo utilizado é implementado pela biblioteca \textit{passlib} em Python, e possui algumas características que aumentam sua segurança. Como pode ser visto na figura \ref{fig:sha256} o algoritmo é utilizado para criptografar a mesma palavra (\textit{password}), porém ele não resulta na mesma palavra criptografada. Isto ocorre pois ele utiliza uma técnica de \textit{salt}, onde é utilizado informações adicionais do sistema para acrescentar uma sequência extra de caracteres no resultado criptografia. Isto o torna mais robusto contra-ataques do tipo \textit{hashtables} \cite{schneier:2007}, que são tabelas que contém palavras mapeadas a sua versão criptografada. Então a senha mestre do usuário é criptografada utilizando o algoritmo SHA256 e a cada tentativa de autenticação sua senha informada é criptografada e então comparada com a armazenada no banco. O algoritmo consegue verificar se elas vêm da mesma palavra mesmo gerando 
\textit{hashs} diferentes.

\begin{figure}[H]
  \centering
  \caption{Algoritmo SHA256.}
  \includegraphics[width=\textwidth]{images/sha256.png}
  \fonte{(AUTORES, 2018)}
  \label{fig:sha256}
\end{figure}

Já para a criptografia das informações de senhas cadastradas pelo usuário, é utilizado uma técnica de criptografia simétrica, pois estas informações precisam ser decifradas quando o usuário deseja visualizá-las. O algoritmo utilizado foi o AES, implementado pela biblioteca \textit{PyCrypto}. O algoritmo utiliza uma chave única, que é utilizada na cifragem e decifragem de textos. Para aumentar a segurança do sistema a chave utilizada no algoritmo é composta de duas partes, uma que é informada pelo usuário durante o seu cadastro, e outra que é mantida pelo sistema. O conjunto das duas gera uma chave única que então é utilizada com o AES. A cada autenticação com sucesso as senhas e informações adicionais do usuário são decifradas e podem ser visualizadas em forma de texto.





\section{Acesso as senhas}
Antes de acessar o sistema é preciso criar um novo usuário, para isto existe uma tela onde são preenchidas suas informações e então criado o usuário. A criação de um usuário não envolve o cadastro e calibragem do reconhecimento facial, pois isto é feito após o primeiro acesso ao sistema. Como pode ser visto na Figura \ref{fig:cadastro-usuario}, no cadastro de um novo usuário é preciso informar um nome de usuário, e-mail, a senha e sua confirmação, assim como uma chave que é utilizada parcialmente para a criptografia das senhas cadastradas futuramente. Também é validado se a senha informada é considerada uma senha forte, contendo caracteres variados como letras maiúsculas e minúsculas, dígitos e caracteres especiais, além de não ser muito curta, pois esta senha é importante para o usuário e não deve ser uma senha simples que pode ser adivinhada através de tentativa e erro.

\begin{figure}[H]
  \centering
  \caption{Cadastro de novo usuário.}
  \includegraphics[width=\textwidth]{images/cadastro-usuario.png}
  \fonte{(AUTORES, 2018)}
  \label{fig:cadastro-usuario}
\end{figure}


Para acessar o sistema é preciso primeiramente ter criado um usuário, então basta utilizar as informações de usuário e senha para acessar o sistema. Na Figura \ref{fig:login} é possível visualizar a tela de login, onde existem dois campos, um para o usuário e outro para a senha, e uma área de visualização previa da câmera do dispositivo. A captura da imagem da câmera serve para o processo de reconhecimento biométrico e para possibilitar o seu uso, é preciso conceder as permissões necessárias para que o usuário possa se autenticar utilizando o reconhecimento facial. Então a autenticação pode ser feita utilizando o usuário e senha cadastrados, ou o usuário e a sua face. Após pressionar o botão entrar, o sistema busca pelo usuário informado, se foi informado uma senha, é feito o login utilizando a mesma, caso contrário é utilizado o reconhecimento biométrico da face capturada no momento em que foi pressionado o botão. Se o usuário e/ou senha estiverem incorretos, ou a face não for reconhecida, o usuário é redirecionado novamente para a tela de login e é apresentado uma mensagem de erro não específica sobre a falha na autenticação.

\begin{figure}[H]
  \centering
  \caption{Tela de login.}
  \includegraphics[width=\textwidth]{images/nao-reconhecida_censored.jpg}
  \fonte{(AUTORES, 2018)}
  \label{fig:login}
\end{figure}






