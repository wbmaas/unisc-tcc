\chapter{Conclusão e Trabalhos Futuros}
\label{conclusao}

O presente trabalho buscou projetar e desenvolver um sistema de gerenciamento de senhas seguro utilizando autenticação biométrica e algoritmos de criptografia. A partir da revisão bibliográfica foi estudado o estado da arte da autenticação biométrica, e identificar quais são suas vantagens e também suas limitações. Também foram pesquisados trabalhos relacionados ao reconhecimento biométrico aplicado em sistemas de gerenciamento de senhas.

Para atingir este objetivo foram estudados trabalhos relacionados onde foram aplicados técnicas de reconhecimento biométrico sobre diversos tipos de biometria. Foram levantados as biometrias e técnicas de reconhecimento utilizadas em cada trabalho e foram comparados seus resultados a fim de definir as biometrias e técnicas a serem aplicadas neste trabalho. Então para o desenvolvimento do sistema de gerenciamento de senhas, foi explorado o uso de características biométricas que podem ser obtidas através de computadores e dispositivos móveis e que possuem bons resultados já comprovados, no caso deste trabalho: faces. 

Foi desenvolvido um sistema web com as funcionalidades de inserção, geração e manipulação de senhas e informações adicionais, que são armazenadas de forma segura utilizando algoritmos de criptografia. Para acessar estas funcionalidades e utilizar o sistema é preciso fazer o cadastro biométrico, e ser autenticado utilizando a face ou a senha mestre. As técnicas de reconhecimento facial utilizadas no sistema demonstram boas taxas de reconhecimento em testes realizados em bases de faces, obtendo aproximadamente 96\% de precisão no reconhecimento.

O sistema desenvolvido apresentou diferentes características presentes nos trabalhos relacionados estudados, e também a soma destas características com novas funcionalidades, que no caso deste trabalho foi o geração e armazenamento seguro de senhas, e a autenticação biométrica por reconhecimento facial.


Para trabalhos futuros, sugere-se explorar outras características biométricas que foram estudadas neste trabalho, porém não foram aplicadas ao sistema, como voz e íris, que possa deixar o sistema ainda mais seguro. Outro ponto a ser estudado é a avaliação da qualidade das leituras biométricas, e atribuição de pesos para cada biometria em um sistema de reconhecimento biométrico multimodal.

Existem diversas técnicas para reconhecimento de características biométricas, como faces e voz. Este trabalho abordou duas destas, \textit{Fisherfaces} e \textit{Eigenfaces}. Trabalhos futuros podem possibilitar de configurar a escolha da técnica a ser utilizada para o reconhecimento de cada biometria.
