\begin{abstract}
\keywords{Gerenciamento de Senhas, Autenticação Biométrica, Criptografia}
  
Milhares de serviços são oferecidos na \textit{web}, e a grande maioria deles utilizam alguma forma de autenticação, geralmente através de usuário e senha. Entretanto, o uso de senhas pode tornar a autenticação frágil quando estas forem muito simples, trazendo pouca segurança. Por outro lado, quando forem complexas e muito grandes são difíceis de memorizar e os usuários muitas vezes guardam anotações destas em locais inseguros, tornando o processo frágil. Desta forma, sistemas de gerenciamento de senhas podem ser uma solução para evitar o uso de senhas fracas e repetidas em diferentes serviços sem a necessidade de serem memorizadas pelo usuário (ou anotadas). De um lado, é necessário utilizar métodos eficientes de autenticação nesses sistemas, pois uma vez que alguém consegue fraudar esse processo, todas as senhas salvas podem ser comprometidas. Por outro lado, é preciso armazenar as senhas de modo seguro utilizando, por exemplo, técnicas de criptografia. Assim, uma boa alternativa é a autenticação biométrica utilizando características como faces e voz, que pode proporcionar uma camada extra de segurança, enquanto melhora a conveniência no seu uso. A maioria dos dispositivos atuais permite o desenvolvimento de sistemas que sejam capazes de processar características biométricas, pois possuem os sensores necessários para o processo de coleta de amostras, através de câmeras, microfones e leitores de digitais e o poder computacional necessário para processá-las. Este trabalho apresenta um sistema de geração de senhas complexas e armazenamento seguro, utilizando criptografia, e autenticação através de reconhecimento facial.
\end{abstract}
