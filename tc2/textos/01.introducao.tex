\chapter{Introdução}
\label{introducao}

O uso de senhas para autenticação é um dos meios mais adotados atualmente, estando presente na maioria dos serviços disponíveis na web, e a utilização de uma única senha para diversos serviços podem trazer vulnerabilidades ao usuário, pois a perda ou descoberta desta senha pode comprometer a segurança de todos os seus sistemas e serviços. Por outro lado, o uso de senhas diferentes para cada serviço acaba se tornando inconveniente pois são difíceis de memorizar \cite{yang:2014}.

Sistemas de armazenamento de senhas já existem atualmente e são bastante utilizadas, muitas vezes estes sistemas acabam sendo funcionalidades de um outro \textit{software}, e acabam não sendo tão seguros como o desejável. Um exemplo disso são navegadores web, que fornecem a opção de salvar senhas e sites favoritos, porém este armazenamento é em sua grande maioria feito em forma de texto simples. Outro ponto é a geração automática de senhas complexas, que é uma funcionalidade simples, porém extremamente subutilizada atualmente. Também existem outras formas de somar segurança neste tipo de sistema, que é utilizando múltiplas formas de autenticação, como autenticação de duas etapas e biometria.

A autenticação através de biometria estabelece a identidade baseada em características físicas e comportamentais (como face e voz), diminuindo a inconveniência para usuários de terem de criar e lembrar de senhas seguras \cite{gofman:2016}. Infelizmente algumas destas características podem ser facilmente obtidas: faces são publicamente disponíveis e digitais permanecem em superfícies intactas. Por isto, o uso de múltiplas características biométricas é preferido a fim de aumentar a segurança e identificação correta do usuário \cite{Agholor:2016}. As características biométricas de uma pessoa podem ser reconhecidas através do uso de técnicas de processamento de imagens e voz aliadas à métodos de aprendizado de máquina. Para isto é necessário construir uma base de dados com exemplos de dados obtidos a partir da leitura destas características, sobre o qual são aplicados os algoritmos de aprendizado de máquina a fim de construir um modelo das propriedades biométricos de cada pessoa \cite{heinen:2005}.

Neste trabalho foi desenvolvido um sistema seguro de armazenamento e geração de senhas, utilizando autenticação biométrica por reconhecimento facial, a fim de melhorar sua usabilidade e segurança. Para a garantir a confiabilidade geral do sistema e realizar o armazenamento de senhas de uma forma segura, foram utilizadas técnicas de criptografia, pois o seu armazenamento em formato de texto simples é uma solução extremamente frágil em termos de segurança. Foram explorados o uso de algoritmos de criptografia baseados em chaves, que podem ser classificados em dois tipos: simétricos, onde geralmente existe uma chave em comum que é utilizada para cifragem e decrifragem; e os de chave pública, onde a chave de cifragem é diferente da de decifragem \cite{schneier:2007}. A vantagem do armazenamento de senhas criptografadas é que mesmo que o acesso a elas seja comprometido elas não podem ser facilmente recuperadas sem que se tenha a chave utilizada na encriptação, e os dados obtidos não passam de sequencias aleatórias de caracteres \cite{chanda:2016}.

A estrutura restante do trabalho está dividida na seguinte forma. O Capítulo 2 descreve um levantamento da fundamentação teórica sobre os assuntos estudados para o desenvolvimento do trabalho. O Capítulo 3 contém uma análise detalhada sobre trabalhos relacionados ao tema de armazenamento de senhas e autenticação biométrica. O Capítulo 4 apresenta a metodologia do estudo aplicado. Já os Capítulos 5, 6 e 7 contém as etapas de modelagem, desenvolvimento e resultados do sistema respectivamente. E por fim o Capítulo 8 apresenta a conclusão e formas de expandir o conhecimento adquirido através de trabalhos futuros.